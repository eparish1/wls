\subsection{\spaceTimeAcronym\ trial subspaces: direct and indirect methods}
We now consider \spaceTimeAcronym\ trial subspaces.  Because the optimization
variables (i.e., the generalized coordinates) in this case are already finite
dimensional, solvers for this type of trial subspace need only develop a
finite-dimensional representation of the objective
functional in problem~\eqref{eq:obj_gen_slab_spacetime}. We describe two
techniques for this purpose: a direct method 
that operates on the FOM O$\Delta$E and an indirect method that operates on the FOM ODE. 

\subsection{\spaceTimeAcronym\ trial subspaces: direct solution approach}
\KTC{Fix this section:
\begin{enumerate} 
\item remove $\theta^\star$
\item replace $J_{D-ST}^n$ in (38) with $J_D^n$. You can do this because of Eq
	(33)
\item remove definition of $J_{D-ST}$.
 \end{enumerate}
}
The direct solution technique seeks to minimize the fully discrete objective
function associated with the FOM O$\Delta$E. We consider linear
multistep methods and leverage the discretization introduced in
Section~\ref{sec:direct}. For notational simplicity, we define an
index-mapping function that is equivalent to the mapping function
$\indexMapper$, but outputs only the first argument: 
$$\indexMappern: (n,i) \mapsto 
\begin{cases}
n & n = 1, \; i = 0, \\
n & n \ge 1, \; i > 0, \\
\indexMappern(n-1,\nstepsArg{n-1}+i) & n > 1, \; i \le 0.
\end{cases}$$
%The definition of the fully discrete objective functions~\eqref{eq:obj_lms} as outlined in Section~\ref{sec:direct}, can be leveraged for this purpose. For simplicity, we only 
%outline the case for linear multistep schemes. 
\methodAcronym\ with an \spaceTimeAcronym\ trial subspace and the direct approach sequentially computes the generalized coordinates $\stgenstateArg{n}$, $n=1,\ldots,\nslabs$ that satisfy
 \begin{equation}\label{eq:obj_gen_lms_final_st}
\begin{split}
& \underset{\stgenstatey \in \RR{\stdimArg{n}}}{\text{minimize } }
\objectiveArgLMS{n}_{\text{D-ST}} (\stbasisArg{n}{\timeWindowArg{n}{1}} \stgenstatey + \stateInterceptSTArg{n},\ldots,\stbasisArg{n}{\timeWindowArg{n}{\nstepsArg{n}}} \stgenstatey + \stateInterceptSTArg{n}),%  \\
%&\text{subject to } \;  \stbasisArg{n}{\timeStartArg{n}}\stgenstateyArg{n} = \begin{cases} 
%\mathbb{P}^n(\stbasisArgnt{n-1}\stgenstateArg{n-1})(\timeEndArg{n-1}) & n = 2,\ldots,\nslabs, \\ 
%\mathbb{P}^n(\stateFOMArgnt{1})(0) &
%n=1. \end{cases}
\end{split} 
\end{equation}
where the space--time objective function for the direct method is defined as
$$
\objectiveArgLMS{n}_{\text{D-ST}}  :  (\stateyDiscreteArgnt{1},\ldots,\stateyDiscreteArgnt{\nstepsArg{n}}) \mapsto \objectiveArgLMS{n}_{\text{D}}(\stateyDiscreteArgnt{1},\ldots,\stateyDiscreteArgnt{\nstepsArg{n}}; \approxstateArg{\indexMappern(n,0)}{\timeWindow^{\indexMapper(n,0)}},\ldots, 
 \approxstateArg{\indexMappern(n,-\lmsWidthArg{n}{1}+1)}{\timeWindow^{\indexMapper(n,-\lmsWidthArg{n}{1}+1)}}).
$$
The boundary conditions are automatically satisfied through the definition of the \spaceTimeAcronym\ trial subspace. The minimization problem~\eqref{eq:obj_gen_lms_final_st} again yields a least-squares problem.
\KTC{I would add a remark that highlights the similarities between the direct
approach for the two kinds of subspaces. They minimize the exact same
objective function, which is interesting.}
\begin{remark}
For the limiting case where one window comprises entire domain (i.e.,
	$\DeltaSlabArg{1}\equiv T$), uniform quadrature weights are used, the trial
	space is set to be $\stspaceSTArg{1} = \stspaceST$, the weighting matrices
	$\stweightingMatArg{1}$ and $\lspgWeightingST$ comprise the identity matrix
	$\mathbf{I}$ \KTC{Don't we just need simply
	$W_{ST}=\text{diag}(W^1)$ (with $W^1$ the half-factor of $A^1$)?. I think this
	would be more general.}, and $\nstepsArg{1} = N_t$ time instances are employed that
	satisfy $\timeWindowArg{1}{i} = t^i$, $i=1,\ldots,N_t$, then
	$\stgenstateArg{1} =  \stgenstate_\text{ST-LSPG}$ and direct \methodAcronym\
	with an \spaceTimeAcronym\ trial subspace recovers ST-LSPG. 
\end{remark}

\begin{remark}
To enable equivalence in the case for a general ST-LSPG weighting matrix
	$\lspgWeightingSTArg{\cdot}$, the weighting matrix $\stweightingMatArg{1}$
	must be time dependent matrix-valued, which associates the objective function 
	\eqref{eq:obj_gen_lms_final_st} with a modified space--time
	norm. For notational simplicity, we do not consider this case in the current
	manuscript.
\end{remark}
 
%\EP{I will rework this section depending on 4.2\\
%Leveraging the definition of the fully discrete objective function~\eqref{eq:obj_lms}, and assuming $N \nstepsArg{n} \ge \stdimArg{n}$, \methodAcronym\ leveraging a \spaceTimeAcronym\ basis sequentially computes the generalized coordinates $\stgenstateArg{n}$, $n=1,\ldots,\nslabs$, that satisfy
% \begin{equation}\label{eq:obj_gen_lms_final_st}
%\begin{split}
%& \underset{\stgenstatey}{\text{minimize } }
%\objectiveArgLMS{n} (\stbasisArg{n}{\timeWindowArg{n}{0}} \stgenstatey + \stateInterceptSTArg{n},\ldots,\stbasisArg{n}{\timeWindowArg{n}{\nstepsArg{n}}} \stgenstatey + \stateInterceptSTArg{n}).%  \\
%%&\text{subject to } \;  \stbasisArg{n}{\timeStartArg{n}}\stgenstateyArg{n} = \begin{cases} 
%%\mathbb{P}^n(\stbasisArgnt{n-1}\stgenstateArg{n-1})(\timeEndArg{n-1}) & n = 2,\ldots,\nslabs, \\ 
%%\mathbb{P}^n(\stateFOMArgnt{1})(0) &
%%n=1. \end{cases}
%\end{split} 
%\end{equation}
%% \begin{equation}\label{eq:obj_gen_lms_final_st}
%%\begin{split}
%%&\stgenstateArg{n} = \underset{\stgenstatey}{\text{arg\,min } }
%%\objectiveArgLMS{n} (\stbasisArg{n}{\timeWindowArg{n}{0}} \stgenstatey + \stateInterceptArg{n},\ldots,\stbasisArg{n}{\timeWindowArg{n}{\nstepsArg{n}}} \stgenstatey + \stateInterceptArg{n})  \\
%%&\text{subject to } \;  \stbasisArg{n}{\timeStartArg{n}}\stgenstateArg{n} = \begin{cases} 
%%\mathbb{P}^n(\stbasisArgnt{n-1}\stgenstateArg{n-1})(\timeEndArg{n-1}) & n = 2,\ldots,\nslabs, \\ 
%%\mathbb{P}^n(\stateFOMArgnt{1})(0) &
%%n=1. \end{cases}\end{split} 
%%\end{equation}
%It is noted that the boundary conditions are automatically satisfied through the definition of the \spaceTimeAcronym\ trial subspace. The minimization problem~\eqref{eq:obj_gen_lms_final_st} can be solved, e.g, via the Gauss--Newton method.
%
%\begin{remark}
%For the limiting case where the window size spans the entire domain, $\DeltaSlabArg{1}\equiv T$, and uniform quadrature weights are used, and the weighting matrix $\stweightingMatArg{n} = \mathbf{I}$, direct \methodAcronym\ with a \spaceTimeAcronym\ trial subspace recovers ST-LSPG with $\lspgWeightingSTArg{\cdot} = \mathbf{I}$. For equivalence conditions in the case where $\lspgWeightingSTArg{\cdot} \ne \mathbf{I}$, the objective function~\eqref{eq:obj_gen_lms_final_st} can be modified to enable weighted space--time norm. 
%\end{remark}
%}
\subsection{\spaceTimeAcronym\ trial subspaces: indirect solution approach}
As opposed to the direct approach, the indirect approach directly minimizes the continuous objective function~\eqref{eq:obj_gen_slab_spacetime} and sequentially computes solutions $\stgenstateArg{n}$, $n = 1,\ldots,\nslabs$ that satisfy
\begin{equation}\label{eq:obj_gen_slab2}
\begin{split}
 & \underset{\stgenstatey \in \RR{\stdimArg{n}}}{\text{minimize }} \mathcal{J}^n \bigg( \stbasisArgnt{n} \stgenstatey + \stateInterceptSTArg{n} \otimes \onesFunctionArg{n} \bigg) .%\\  
%&\text{subject to } \;  \stbasisArg{n}{\timeStartArg{n}}\stgenstateyArg{n} = \begin{cases} 
%\mathbb{P}^n(\stbasisArgnt{n-1}\stgenstateArg{n-1})(\timeEndArg{n-1}) & n = 2,\ldots,\nslabs, \\ 
%\mathbb{P}^n(\stateFOMArgnt{1})(0) &
%n=1. \end{cases} 
\end{split} 
\end{equation}
%\begin{equation}\label{eq:obj_gen_slab2}
%\begin{split}
% &\stgenstate^n =
%\underset{\stgenstatey}{\text{arg\,min }} \mathcal{J}^n \bigg( \stbasisArgnt{n} \stgenstatey + \stateInterceptArg{n}  \bigg) \\  
%&\text{subject to } \;  \stbasisArg{n}{\timeStartArg{n}}\stgenstateArg{n} = \begin{cases} 
%\mathbb{P}^n(\stbasisArgnt{n-1}\stgenstateArg{n-1})(\timeEndArg{n-1}) & n = 2,\ldots,\nslabs, \\ 
%\mathbb{P}^n(\stateFOMArgnt{1})(0) &
%n=1. \end{cases} 
%\end{split} 
%\end{equation}
%The first order optimality conditions are given by,
%\begin{equation}\label{eq:st_stationary}
% \intSlabArg{n} \bigg[ \stbasisDotArg{n}{t}^T  - \stbasisArg{n}{t}^T \bigg[\frac{\partial
%\velocity}{\partial \stateyDiscrete} (\stbasisDotArg{n}{t} \stgenstateArg{n} +                    
%\stateInterceptArg{n})\bigg]^T  \bigg] \stweightingMatArg{n} \bigg( \stbasisDotArg{n}{t} \stgenstateArg{n}  - \velocity (\stbasisArg{n}{t} \stgenstateArg{n} + \stateInterceptArg{n}) \bigg) dt = \bz.\end{equation}
Numerically solving the minimization problem requires the introduction of a quadrature rule for 
discretization of the integral. To this end, we introduce $\ncollocSTArg{n} \ge \text{ceil}(\stdimArg{n}/ \text{rank}(\stweightingMatOneArg{n}))$ quadrature points over the $n$th window, $ \{ \collocPointSTArg{n}{i} \}_{i=1}^{\ncollocSTArg{n}} \subset [\timeStartArg{n},\timeEndArg{n}]$, $n=1,\ldots,\nslabs$. 
%with $\timeStartArg{n} \le \collocPointSTArg{n}{1} \le \cdots \le \collocPointSTArg{n}{\ncollocSTArg{n}} \le \timeEndArg{n}.$
Leveraging these quadrature points, \methodAcronym\ with the indirect method and an \spaceTimeAcronym\ trial subspace computes the generalized coordinates 
$\stgenstateArg{n}$, $n=1,\ldots,\nslabs$ that satisfy
\begin{equation}\label{eq:obj_gen_slab2} 
\begin{split}
&\underset{\stgenstatey \in \RR{\stdimArg{n}}}{\text{minimize }} \objectiveDiscreteSTArg{n} \bigg( \stbasisArgnt{n} \stgenstatey + \stateInterceptSTArg{n} \otimes \onesFunctionArg{n}  \bigg) , %\\
%&\text{subject to }\;  \stbasisArg{n}{\timeStartArg{n}}\stgenstateyArg{n} = \begin{cases} 
%\mathbb{P}^n(\stbasisArgnt{n-1}\stgenstateArg{n-1})(\timeEndArg{n-1}) & n = 2,\ldots,\nslabs, \\ 
%\mathbb{P}^n(\stateFOMArgnt{1})(0) &
%n=1,
%\end{cases}
\end{split} 
\end{equation}
%\begin{equation}\label{eq:obj_gen_slab2} 
%\begin{split}&\stgenstate^n =
%\underset{\stgenstatey}{\text{arg\,min }} \objectiveDiscreteSTArg{n} \bigg( \stbasisArgnt{n} \stgenstatey + \stateInterceptArg{n}  \bigg) , \\
%&\text{subject to }\;  \stbasisArg{n}{\timeStartArg{n}}\stgenstateArg{n} = \begin{cases} 
%\mathbb{P}^n(\stbasisArgnt{n-1}\stgenstateArg{n-1})(\timeEndArg{n-1}) & n = 2,\ldots,\nslabs, \\ 
%\mathbb{P}^n(\stateFOMArgnt{1})(0) &
%n=1,
%\end{cases}
%\end{split} 
%\end{equation}
where the discrete objective function is given by
$$\objectiveDiscreteSTArg{n}: \statey \mapsto \frac{1}{2}\sum_{i=1}^{\ncollocSTArg{n}} \zeta^{n,i} 
\bigg[ \dot{\statey}(\collocPointSTArg{n}{i})  - \velocity (\stateyArg{}{\collocPointSTArg{n}{i}},\collocPointSTArg{n}{i} ) \bigg]^T 
\stweightingMatArg{n} 
\bigg[ \dot{\statey}(\collocPointSTArg{n}{i})  - \velocity (\stateyArg{}{\collocPointSTArg{n}{i}},\collocPointSTArg{n}{i} ) \bigg]
$$
\KTC{Can you add the domain and codomain of the function to be consistent with
earlier functionals?} and $\zeta^{n,i} \in \RRplus$, $i=1,\ldots,\ncollocSTArg{n}$ are quadrature
weights over the $n$th time window. The optimization problem~\eqref{eq:obj_gen_slab2} again comprises a least-squares problem.
\KTC{It would be great to a comment about how this is equivalent to Qiqi Wang
and Paul Constantine's method in the case of one time window and $\Pi^1$
corresponding to full solution trajectories. They use collocation-based
hyper-reduction at a few points in time (but for the full ODE residual at that
time), which you could mention. This is a very cool connection I didn't
realize before, and I think it's definitely worth mentioning!}
%$$\objectiveDiscreteSTArg{n}: \stgenstatey \mapsto \sum_{i=1}^{\ncollocSTArg{n}} \gamma_i 
%\bigg[ \stbasisDotArg{n}{\collocPointSTArg{n}{i}} \stgenstateyArg{ }  - \velocity (\stbasisArg{n}{\collocPointSTArg{n}{i}} \stgenstateyArg{ } + \stateInterceptSTArg{n},\collocPointSTArg{n,i}) \bigg]^T 
%\stweightingMatArg{n} 
%\bigg[ \stbasisDotArg{n}{\collocPointSTArg{n}{i}} \stgenstateyArg{ }  - \velocity (\stbasisArg{n}{\collocPointSTArg{n}{i}} \stgenstateyArg{ } + \stateInterceptSTArg{n},\collocPointSTArg{n,i}) \bigg].
%$$
%The stationary points are given by,
%\begin{multline*}\label{eq:st_stationary_discrete}
%\frac{\partial \objectiveDiscreteSTArg{n}}{\partial \stgenstatey}(\stgenstateArg{n}) = 
%\sum_{i=0}^{\ncollocSTArg{n}} \gamma_i \bigg[ \stbasisDotArg{n}{\collocPointSTArg{n}{i}}^T  - \stbasisArg{n}{\collocPointSTArg{n}{i}}^T \bigg[\frac{\partial
%\velocity}{\partial \stateyDiscrete} (\stbasisDotArg{n}{\collocPointSTArg{n}{i}} \stgenstateArg{n} +                    
%\stateInterceptArg{n})\bigg]^T  \bigg] \stweightingMatArg{n} \bigg( \stbasisDotArg{n}{\collocPointSTArg{n}{i}} \stgenstateArg{n}  - \velocity (\stbasisArg{n}{\collocPointSTArg{n}{i}} \stgenstateArg{n} + \stateInterceptArg{n}) \bigg),\end{multline*}
%%subject to the boundary conditions 
%\begin{equation}
%\stbasisArg{n}{\timeStartArg{n}}\stgenstateArg{n} = \begin{cases} 
%\stbasisArg{n-1}{\timeEndArg{n-1}}\stgenstateArg{n-1}  & n = 2,\ldots,\nslabs, \\ 
%\approxstateIC &
%n=1. \end{cases} \end{equation} 
\subsection{\spaceTimeAcronym\ trial subspaces: summary}
\spaceTimeAcronym\ trial subspaces yield a series of space--time systems of algebraic equations over each window. As a variety of work has examined space--time reduced-order models with \spaceTimeAcronym\ trial subspaces, 
a detailed exposition of solution techniques for these systems is not pursued here. It is sufficient to say that the 
space--time trial subspace yields a series of dense systems to be solved over each window.

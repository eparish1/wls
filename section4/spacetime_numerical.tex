\subsection{\spaceTimeAcronym\ Trial Spaces: Direct and Indirect Methods}
We now briefly consider \spaceTimeAcronym\ trial subspaces. 
For \spaceTimeAcronym\ trial subspaces, the distinction between a direct and indirect method is less clear as the optimization variable (i.e., the generalized coordinates) are finite 
dimensional. There is no need to transcribe an infinite dimensional optimization variable into a finite dimensional one. There is, however, still a need to develop a finite 
dimensional representation of the objective function~\eqref{eq:obj_gen_slab}. In what follows, we describe two techniques to do this: one that works with the FOM O$\Delta$E and one that works with the FOM ODE. We associate direct methods as those 
that work with the O$\Delta$E and indirect methods with those that work directly with the continuous ODE.  

\subsection{\spaceTimeAcronym\ Trial Spaces: Direct Solution Technique}
The direct solution technique seeks to minimize the fully discrete objective function associated with the O$\Delta$E. For brevity, we again focus on linear multistep methods. 

%The definition of the fully discrete objective functions~\eqref{eq:obj_lms} as outlined in Section~\ref{sec:direct}, can be leveraged for this purpose. For simplicity, we only 
%outline the case for linear multistep schemes. 

Leveraging the definition of the fully discrete objective function~\eqref{eq:obj_lms}, and assuming $N \nstepsArg{n} \ge \stdimArg{n}$, \methodAcronym\ leveraging a \spaceTimeAcronym\ basis is given by, 
 \begin{equation}\label{eq:obj_gen_lms_final_st}
\begin{split}
& \underset{\stgenstatey}{\text{minimize } }
\objectiveArgLMS{n} (\stbasisArg{n}{\timeWindowArg{n}{0}} \stgenstatey + \stateInterceptArg{n},\ldots,\stbasisArg{n}{\timeWindowArg{n}{\nstepsArg{n}}} \stgenstatey + \stateInterceptArg{n})  \\
&\text{subject to } \;  \stbasisArg{n}{\timeStartArg{n}}\stgenstateyArg{n} = \begin{cases} 
\mathbb{P}^n(\stbasisArgnt{n-1}\stgenstateArg{n-1})(\timeEndArg{n-1}) & n = 2,\ldots,\nslabs, \\ 
\mathbb{P}^n(\stateFOMArgnt{1})(0) &
n=1. \end{cases}\end{split} 
\end{equation}

% \begin{equation}\label{eq:obj_gen_lms_final_st}
%\begin{split}
%&\stgenstateArg{n} = \underset{\stgenstatey}{\text{arg\,min } }
%\objectiveArgLMS{n} (\stbasisArg{n}{\timeWindowArg{n}{0}} \stgenstatey + \stateInterceptArg{n},\ldots,\stbasisArg{n}{\timeWindowArg{n}{\nstepsArg{n}}} \stgenstatey + \stateInterceptArg{n})  \\
%&\text{subject to } \;  \stbasisArg{n}{\timeStartArg{n}}\stgenstateArg{n} = \begin{cases} 
%\mathbb{P}^n(\stbasisArgnt{n-1}\stgenstateArg{n-1})(\timeEndArg{n-1}) & n = 2,\ldots,\nslabs, \\ 
%\mathbb{P}^n(\stateFOMArgnt{1})(0) &
%n=1. \end{cases}\end{split} 
%\end{equation}
 
The minimization problem~\eqref{eq:obj_gen_lms_final_st} can be solved, e.g, via the Gauss-Newton method.

\begin{remark}
For the limiting case where the window size spans the entire domain, $\DeltaSlabArg{1}\equiv T$, direct \methodAcronym\ with a \spaceTimeAcronym\ trial subspace recovers ST-LSPG.
\end{remark}

\subsection{\spaceTimeAcronym\ Trial Spaces: Indirect Solution Technique}
As opposed to the direct solution technique, the indirect solution technique directly minimizes the continuous objective function~\eqref{eq:obj_gen_slab_spacetime},
\begin{equation}\label{eq:obj_gen_slab2}
\begin{split}
 & \underset{\stgenstatey}{\text{minimize }} \mathcal{J}^n \bigg( \stbasisArgnt{n} \stgenstatey + \stateInterceptArg{n}  \bigg) \\  
&\text{subject to } \;  \stbasisArg{n}{\timeStartArg{n}}\stgenstateyArg{n} = \begin{cases} 
\mathbb{P}^n(\stbasisArgnt{n-1}\stgenstateArg{n-1})(\timeEndArg{n-1}) & n = 2,\ldots,\nslabs, \\ 
\mathbb{P}^n(\stateFOMArgnt{1})(0) &
n=1. \end{cases} 
\end{split} 
\end{equation}
%\begin{equation}\label{eq:obj_gen_slab2}
%\begin{split}
% &\stgenstate^n =
%\underset{\stgenstatey}{\text{arg\,min }} \mathcal{J}^n \bigg( \stbasisArgnt{n} \stgenstatey + \stateInterceptArg{n}  \bigg) \\  
%&\text{subject to } \;  \stbasisArg{n}{\timeStartArg{n}}\stgenstateArg{n} = \begin{cases} 
%\mathbb{P}^n(\stbasisArgnt{n-1}\stgenstateArg{n-1})(\timeEndArg{n-1}) & n = 2,\ldots,\nslabs, \\ 
%\mathbb{P}^n(\stateFOMArgnt{1})(0) &
%n=1. \end{cases} 
%\end{split} 
%\end{equation}
%The first order optimality conditions are given by,
%\begin{equation}\label{eq:st_stationary}
% \intSlabArg{n} \bigg[ \stbasisDotArg{n}{t}^T  - \stbasisArg{n}{t}^T \bigg[\frac{\partial
%\velocity}{\partial \stateyDiscrete} (\stbasisDotArg{n}{t} \stgenstateArg{n} +                    
%\stateInterceptArg{n})\bigg]^T  \bigg] \stweightingMatArg{n} \bigg( \stbasisDotArg{n}{t} \stgenstateArg{n}  - \velocity (\stbasisArg{n}{t} \stgenstateArg{n} + \stateInterceptArg{n}) \bigg) dt = \bz.\end{equation}
Numerically solving the minimization problem requires the introduction of a quadrature rule for 
discretization of the integral. We introduce $\ncollocSTArg{n} \ge \text{ciel}(\stdim / N )$ collocation points over the $n$th window, 
$$\timeStartArg{n} \le \collocPointSTArg{n}{1} \le \ldots \le \collocPointSTArg{n}{\ncollocSTArg{n}} \le \timeEndArg{n}.$$
Leveraging these collocation points, the indirect method for \methodAcronym\ with a \spaceTimeAcronym\ trial subspace yields,
\begin{equation}\label{eq:obj_gen_slab2} 
\begin{split}
&\underset{\stgenstatey}{\text{minimize }} \objectiveDiscreteSTArg{n} \bigg( \stbasisArgnt{n} \stgenstatey + \stateInterceptArg{n}  \bigg) , \\
&\text{subject to }\;  \stbasisArg{n}{\timeStartArg{n}}\stgenstateyArg{n} = \begin{cases} 
\mathbb{P}^n(\stbasisArgnt{n-1}\stgenstateArg{n-1})(\timeEndArg{n-1}) & n = 2,\ldots,\nslabs, \\ 
\mathbb{P}^n(\stateFOMArgnt{1})(0) &
n=1,
\end{cases}
\end{split} 
\end{equation}
%\begin{equation}\label{eq:obj_gen_slab2} 
%\begin{split}&\stgenstate^n =
%\underset{\stgenstatey}{\text{arg\,min }} \objectiveDiscreteSTArg{n} \bigg( \stbasisArgnt{n} \stgenstatey + \stateInterceptArg{n}  \bigg) , \\
%&\text{subject to }\;  \stbasisArg{n}{\timeStartArg{n}}\stgenstateArg{n} = \begin{cases} 
%\mathbb{P}^n(\stbasisArgnt{n-1}\stgenstateArg{n-1})(\timeEndArg{n-1}) & n = 2,\ldots,\nslabs, \\ 
%\mathbb{P}^n(\stateFOMArgnt{1})(0) &
%n=1,
%\end{cases}
%\end{split} 
%\end{equation}
where the discrete objective function is given by, 
$$\objectiveDiscreteSTArg{n}: \stgenstatey \mapsto \sum_{i=1}^{\ncollocSTArg{n}} \gamma_i 
\bigg[ \stbasisDotArg{n}{\collocPointSTArg{n}{i}} \stgenstateyArg{ }  - \velocity (\stbasisArg{n}{\collocPointSTArg{n}{i}} \stgenstateyArg{ } + \stateInterceptArg{n},\collocPointSTArg{n,i}) \bigg]^T 
\stweightingMatArg{n} 
\bigg[ \stbasisDotArg{n}{\collocPointSTArg{n}{i}} \stgenstateyArg{ }  - \velocity (\stbasisArg{n}{\collocPointSTArg{n}{i}} \stgenstateyArg{ } + \stateInterceptArg{n},\collocPointSTArg{n,i}) \bigg].
$$
The nonlinear least-squares problem can again be solved, e.g., via the Gauss-Newton method. 
%The stationary points are given by,
%\begin{multline*}\label{eq:st_stationary_discrete}
%\frac{\partial \objectiveDiscreteSTArg{n}}{\partial \stgenstatey}(\stgenstateArg{n}) = 
%\sum_{i=0}^{\ncollocSTArg{n}} \gamma_i \bigg[ \stbasisDotArg{n}{\collocPointSTArg{n}{i}}^T  - \stbasisArg{n}{\collocPointSTArg{n}{i}}^T \bigg[\frac{\partial
%\velocity}{\partial \stateyDiscrete} (\stbasisDotArg{n}{\collocPointSTArg{n}{i}} \stgenstateArg{n} +                    
%\stateInterceptArg{n})\bigg]^T  \bigg] \stweightingMatArg{n} \bigg( \stbasisDotArg{n}{\collocPointSTArg{n}{i}} \stgenstateArg{n}  - \velocity (\stbasisArg{n}{\collocPointSTArg{n}{i}} \stgenstateArg{n} + \stateInterceptArg{n}) \bigg),\end{multline*}
%%subject to the boundary conditions 
%\begin{equation}
%\stbasisArg{n}{\timeStartArg{n}}\stgenstateArg{n} = \begin{cases} 
%\stbasisArg{n-1}{\timeEndArg{n-1}}\stgenstateArg{n-1}  & n = 2,\ldots,\nslabs, \\ 
%\approxstateIC &
%n=1. \end{cases} \end{equation} 

\subsection{\spaceTimeAcronym\ Trial Space: Summary}
As was just shown, \spaceTimeAcronym\ trial subspaces yield a series of space--time algebraic systems over each window. As a variety of work has examined space--time reduced-order models with \spaceTimeAcronym\ trial subspaces, 
a detailed exposition of solution techniques for these systems is not pursued here. It is sufficient to say that the 
space--time trial subspace yields a series of dense (non)linear systems to be solved over each window.

\subsection{\spaceTimeAcronym\ Trial Spaces}
We now briefly consider \spaceTimeAcronym\ trial spaces. Over 
each window, we introduce the space--time trial space
$$ \stateArgnt{n} \in \stspaceSTArg{n} \subseteq \RR{N} \otimes \timeSpaceArg{n},$$
along with the space--time trial basis matrix \textit{function}, 
\begin{align*}
\stbasisArgnt{n} &: t \mapsto \stbasis(t) \\
 &: [\timeStartArg{n},\timeEndArg{n}] \rightarrow  \RR{N \times \stdimArg{n}}  ,%\spaceTrialSpace , \\
\end{align*}
with $\text{Range}(\stbasisArgnt{n}) + \stateInterceptArg{} \equiv \stspaceSTArg{n} $ and where $\stdimArg{n}$ is the number of space--time generalized coordinates over the $n$th window. 
For simplicity we assume the reference state to be equivalent for each time window, although no such requirement is necessary. 
The state over each window is approximated by,
\begin{equation}\label{eq:stapprox}
 \stateFOMArg{n}{t} \approx \approxstateArg{n}{t}  = \stbasisArg{n}{t} \stgenstateArg{n} + \stateInterceptArg{},
\end{equation}
%and the state is then approximated by,
%\begin{equation}\label{eq:stapprox}
% \stateArg{n}{t} \approx \approxstateArg{n}{t}  = \stbasisArg{n}{t} \stgenstate + \stateInterceptArg{n},
%\end{equation}
where $\stgenstateArg{n} \in \RR{\stdimArg{n}}$ are the space--time generalized coordinates over the $n$th window.


Substituting the approximation~\eqref{eq:stapprox} into the minimization problem~\eqref{eq:tclsrm}, \methodAcronym\ with 
the \spaceTimeAcronym\ trial space computes the sequence of generalized coordinates $\stgenstateArg{n}$, for $n=1,\ldots,\nslabs$, as the solution to the minimization problems,
\begin{align}\label{eq:obj_gen_slab_spacetime}
\begin{split}
&\stgenstateArg{n} = \underset{\stgenstatey \in \RR{\stdim}}{\text{arg\,min}}\; \mathcal{J}^n( \stbasisArgnt{n} \stgenstatey + \stateInterceptArg{n} ), \\ 
      &\text{subject to }\;  \stbasisArg{n}{\timeStartArg{n}}\stgenstateArg{n}  =
  \begin{cases} 
\mathbb{P}^n(\stbasisArgnt{n-1}\stgenstateArg{n-1})(\timeEndArg{n-1})  & n = 2,\ldots,\nslabs, \\ 
\mathbb{P}^n(\stateFOMArgnt{1})(0)
 & n=1. \end{cases}
%\begin{cases} \genstateyArg{n-1}{\timeEndArg{n-1}} & n = 2,\ldots,\nslabs \\
%\genstateIC & n=1, \end{cases} 
\end{split}
\end{align}

%Inserting the approximation~\eqref{eq:stapprox} into the optimization problem~\eqref{eq:tclsrm}, the
%constrained minimization problem over the $n$th time slab can be re-written as,
%\begin{equation}\label{eq:obj_gen_slab_spacetime} \stgenstate^n =
%\underset{\stgenstatey}{\text{argmin }} \mathcal{J}^n \bigg( \stbasisArgnt{n} \stgenstatey + \stateInterceptArg{n}  \bigg) , \end{equation}
%%\begin{equation}\label{eq:obj_gen_slab} \genstate^n(t) =
%%\underset{\genstatey}{\text{argmin }} \mathcal{J}^n
%%\bigg(\decoder\big(\genstatey(\tau)\big) \bigg) , \end{equation}
%subject to the boundary conditions 
%\begin{equation}\label{eq:bcs_spacetime}
%\stbasisArg{n}{\timeStartArg{n}}\stgenstateArg{n} = \begin{cases} 
%\mathbb{P}^n\stbasisArg{n-1}{\timeEndArg{n-1}}\stgenstateArg{n-1}  & n = 2,\ldots,\nslabs, \\ 
%\mathbb{P}^n \stateFOMIC
% & n=1. \end{cases} \end{equation}
\subsubsection{Stationary Conditions} 
The key difference between \spaceTimeAcronym\ and \spatialAcronym\ trial spaces is as follows: in \spaceTimeAcronym\ trial spaces the generalized coordinates comprise a vector in $\RR{\stdimArg{n}}$, while in the \spatialAcronym\ trial spaces the generalized coordinates comprise a \textit{vector of functions}, i.e., the generalized coordinates live in $\RR{\romdim} \otimes \timeSpaceArg{n}$. 
Thus, in the \spaceTimeAcronym\ case the optimization problem is no longer minimizing a functional with respect to a function, but now is minimizing a function with 
respect to vector.  As such, the first-order optimality conditions can be derived using standard calculus. Differentiating the objective with respect to $\stgenstatey$ and setting equal to zero yields,
\begin{equation}\label{eq:st_stationary}
 \intSlabArg{n} \bigg[ \stbasisDotArg{n}{t}^T  - \stbasisArg{n}{t}^T \bigg[\frac{\partial
\velocity}{\partial \stateyDiscrete} (\stbasisDotArg{n}{t} \stgenstateArg{n} +                    
\stateInterceptArg{n},t)\bigg]^T  \bigg] \stweightingMatArg{n} \bigg( \stbasisDotArg{n}{t} \stgenstateArg{n}  - \velocity (\stbasisArg{n}{t} \stgenstateArg{n} + \stateInterceptArg{n},t) \bigg) dt = \bz.\end{equation}
%subject to the boundary conditions 
%\begin{equation}\label{eq:bcs_spacetime}
%\stbasisArg{n}{\timeStartArg{n}}\stgenstateArg{n} = \begin{cases} 
%\mathbb{P}^n\stbasisArg{n-1}{\timeEndArg{n-1}}\stgenstateArg{n-1}  & n = 2,\ldots,\nslabs, \\ 
%\mathbb{P}^n \stateFOMIC &
%n=1. \end{cases} \end{equation}
For \spaceTimeAcronym\ trial spaces, the stationary conditions comprise an algebraic system of integral equations, as opposed to a system of differential equations.

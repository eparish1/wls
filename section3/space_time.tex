\subsection{\spaceTimeAcronym\ trial spaces}\label{sec:wls_spacetime}
The \spaceTimeAcronym\ trial subspace over the $n$th time window approximates
the FOM ODE solution trajectory $\stateFOM\in\RR{N}\otimes\timeSpace$
with $\approxstateArgnt{n} \in \stspaceSTArg{n}$, where
\begin{equation}\label{eq:st_sttrialspace}
 \stspaceSTArg{n} \defeq
 \Range{\stbasisArgnt{n}} + \stateInterceptSTArg{n}\otimes \onesFunctionArg{n} \subseteq \RR{\fomdim} \otimes \timeSpaceArg{n}.
\end{equation}
Here $\stbasisArgnt{n} \in \RR{\fomdim \times
\stdimArg{n}}\otimes\timeSpaceArg{n}$, $n=1,\ldots,\nslabs$, with $\stbasisArgnt{n}: \timeDummy \mapsto \stbasisArgnt{n}(\timeDummy)$ and $\stbasisArg{n}{\timeStartArg{n}} = \bz$ is the space--time trial basis matrix {function} and $\stateInterceptSTArg{n} \in \RR{\fomdim}$, $n=1,\ldots,\nslabs$ provides the affine transformation. 
To enforce the initial condition and ensure solution continuity across time windows, we set $\stateInterceptSTArg{1} = \stateFOMIC$ and
$\stateInterceptSTArg{n} = \approxstateArg{n-1}{\timeEndArg{n-1}}$ for
$n=2,\ldots,\nslabs$.
At any time instance $t \in [\timeStartArg{n},\timeEndArg{n}]$, the \spaceTimeAcronym\ trial subspace approximates the FOM ODE solution as
 \begin{equation}\label{eq:stapprox}
 \stateFOMArg{n}{t} \approx \approxstateArg{n}{t}  = \stbasisArg{n}{t} \stgenstateArg{n} + \stateInterceptSTArg{n},
\end{equation}
where $\stgenstateArg{n} \in \RR{\stdimArg{n}}$ are the space--time generalized coordinates over the $n$th window. 
Setting $\stspace^n \leftarrow \stspaceSTArg{n}$ in the
\methodAcronym\ minimization problem~\eqref{eq:tclsrm} implies that
\methodAcronym\ with \spaceTimeAcronym\ trial subspaces sequentially computes
solutions $\stgenstateArg{n}$, $n = 1,\ldots,\nslabs$ that satisfy
\begin{align}\label{eq:obj_gen_slab_spacetime}
\begin{split}
&\underset{\stgenstateyArg{} \in \RR{\stdimArg{n}}}{\text{minimize}}\; \mathcal{J}^n( \stbasisArgnt{n} \stgenstateyArg{} + \stateInterceptSTArg{n} \otimes \onesFunctionArg{n} ).% \\ 
%      &\text{subject to }\;  \stbasisArg{n}{\timeStartArg{n}}\stgenstateyArg{n}  =0
%  \begin{cases} 
%\mathbb{P}^n(\stbasisArgnt{n-1}(\timeEndArg{n-1})\stgenstateArg{n-1})  & n = 2,\ldots,\nslabs, \\ 
%\mathbb{P}^n(\stateFOMIC)
% & n=1. \end{cases}
%%\begin{cases} \genstateyArg{n-1}{\timeEndArg{n-1}} & n = 2,\ldots,\nslabs \\
%\genstateIC & n=1, \end{cases} 
\end{split}
\end{align}
%\begin{align}\label{eq:obj_gen_slab_spacetime}
%\begin{split}
%&\stgenstateArg{n} = \underset{\stgenstatey \in \RR{\stdim}}{\text{arg\,min}}\; \mathcal{J}^n( \stbasisArgnt{n} \stgenstatey + \stateInterceptArg{n} ), \\ 
%      &\text{subject to }\;  \stbasisArg{n}{\timeStartArg{n}}\stgenstateArg{n}  =
%  \begin{cases} 
%\mathbb{P}^n(\stbasisArgnt{n-1}\stgenstateArg{n-1})(\timeEndArg{n-1})  & n = 2,\ldots,\nslabs, \\ 
%\mathbb{P}^n(\stateFOMArgnt{1})(0)
% & n=1. \end{cases}
%%\begin{cases} \genstateyArg{n-1}{\timeEndArg{n-1}} & n = 2,\ldots,\nslabs \\
%%\genstateIC & n=1, \end{cases} 
%\end{split}
%\end{align}

%Inserting the approximation~\eqref{eq:stapprox} into the optimization problem~\eqref{eq:tclsrm}, the
%constrained minimization problem over the $n$th time slab can be re-written as,
%\begin{equation}\label{eq:obj_gen_slab_spacetime} \stgenstate^n =
%\underset{\stgenstatey}{\text{argmin }} \mathcal{J}^n \bigg( \stbasisArgnt{n} \stgenstatey + \stateInterceptArg{n}  \bigg) , \end{equation}
%%\begin{equation}\label{eq:obj_gen_slab} \genstate^n(t) =
%%\underset{\genstatey}{\text{argmin }} \mathcal{J}^n
%%\bigg(\decoder\big(\genstatey(\tau)\big) \bigg) , \end{equation}
%subject to the boundary conditions 
%\begin{equation}\label{eq:bcs_spacetime}
%\stbasisArg{n}{\timeStartArg{n}}\stgenstateArg{n} = \begin{cases} 
%\mathbb{P}^n\stbasisArg{n-1}{\timeEndArg{n-1}}\stgenstateArg{n-1}  & n = 2,\ldots,\nslabs, \\ 
%\mathbb{P}^n \stateFOMIC
% & n=1. \end{cases} \end{equation}
\subsubsection{Stationary conditions} 
The key difference between \spaceTimeAcronym\ and \spatialAcronym\ trial
	subspaces is as follows: generalized coordinates for \spaceTimeAcronym\ trial subspaces 
	comprise a vector in $\RR{\stdimArg{n}}$, while generalized coordinates for
	the \spatialAcronym\ trial subspaces comprise a
	time-dependent vector in
	$\RR{\romdimArg{n}} \otimes \timeSpaceArg{n}$. 
Thus, in the \spaceTimeAcronym\ case, the optimization problem is no longer
	minimizing a functional with respect to a (time-dependent) function, but is minimizing a function with 
respect to a vector.  As such, the first-order optimality conditions can be
	derived using standard calculus. Differentiating the objective function with respect
	to the generalized coordinates and setting the result equal to zero yields 
\begin{equation}\label{eq:st_stationary}
 \intSlabArg{n} \bigg[ \stbasisDotArg{n}{t}^T  - \stbasisArg{n}{t}^T \bigg[\frac{\partial
\velocity}{\partial \stateyDiscrete} (\stbasisDotArg{n}{t} \stgenstateArg{n} +                    
\stateInterceptSTArg{n},t)\bigg]^T  \bigg] \stweightingMatArg{n} \bigg( \stbasisDotArg{n}{t} \stgenstateArg{n}  - \velocity (\stbasisArg{n}{t} \stgenstateArg{n} + \stateInterceptSTArg{n},t) \bigg) dt = \bz.\end{equation}
%subject to the boundary conditions 
%\begin{equation}\label{eq:bcs_spacetime}
%\stbasisArg{n}{\timeStartArg{n}}\stgenstateArg{n} = \begin{cases} 
%\mathbb{P}^n\stbasisArg{n-1}{\timeEndArg{n-1}}\stgenstateArg{n-1}  & n = 2,\ldots,\nslabs, \\ 
%\mathbb{P}^n \stateFOMIC &
%n=1. \end{cases} \end{equation}
Thus, for \spaceTimeAcronym\ trial subspaces, the stationary conditions
	comprise a system of algebraic equations, as opposed to a system of
	differential equations as with \spatialAcronym\ trial subspaces.

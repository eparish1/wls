\subsection{\spatialAcronym\ trial spaces}
The \spatialAcronym\ trial subspace over the $n$th time window is given as, 
\begin{equation}\label{eq:sttrialspace}
 \approxstateArgnt{n} \in \stspaceSArg{n} \defeq 
\trialspace \otimes \timeSpaceArg{n} +
	\stateIntercept\otimes\onesFunction^n,\\ 
%\trialspace \otimes \timeSpaceArg{n},
\end{equation}
where the spatial trial subspace, $\trialspace$, is as defined in~\eqref{eq:spatial_subspace} and 
$\onesFunction^n\in\timeSpaceArg{n}$ is defined as $\onesFunction:[\timeStartArg{n},\timeEndArg{n}]\rightarrow\{1\}.$
% and  
%$\approxstateArgnt{n}: [\timeStartArg{n},\timeEndArg{n}] \rightarrow  \trialspace$ is the \methodAcronym\ approximation to the state over the $n$th time slab.
For simplicity, we study the case 
where the spatial trial subspace is equivalent over each time window. Over the $n$th time window, the state is approximated within the trial subspace as, 
\begin{equation}\label{eq:affine_trialspace_tclsrm}
\stateFOMArg{n}{t} \approx \approxstateArgnt{n}(t) \defeq \basisspace \genstateArg{n}{t} + \stateIntercept,
\end{equation}
where $\stateFOMArgnt{n}: [\timeStartArg{n},\timeEndArg{n}] \rightarrow \mathbb{R}^{\fomdim}$ is the FOM state over the $n$th window, and $\genstateArgnt{n} \in \RR{\romdim} \otimes \timeSpaceArg{n}$ with $\genstateArgnt{n}: [\timeStartArg{n},\timeEndArg{n}] \rightarrow \mathbb{R}^{\romdim}$ are the generalized coordinates over the $n$th time window. 
Substituting the approximation~\eqref{eq:affine_trialspace_tclsrm} into the minimization problem~\eqref{eq:tclsrm}, \methodAcronym\ computes  
the generalized coordinates $\genstateArgnt{n}$, for $n = 1,\ldots,\nslabs$, as the solution to the minimization problems,
%For notational purposes, we define the ``decoder", or ``lifting operator",
%that maps the generalized coordinates to the state-space of the full-order
%model as, \begin{align}\label{eq:decoder} \decoder  \vcentcolon & \;
%\genstatey(t) \mapsto \basisspace \genstatey(t) + \stateIntercept, \\
%\vcentcolon & \; \RR{\romdim}  \rightarrow \RR{\fomdim} .  \end{align}
%With $\state(t) \approx \approxstate(t)  =  \decoder(\genstate(t))$, the
%constrained minimization problem at the $n$th time slab can be re-written as,
%The
%constrained minimization problem over the $n$th time slab can be re-written as,

\begin{equation}\label{eq:obj_gen_slab}
\begin{split}
      & \underset{\genstateyArgnt{n} \in \RR{\romdim}}{\text{minimize}}\; \mathcal{J}^n(\basisspace \genstateyArgnt{n} + \stateIntercept), \\ 
      & \text{subject to }\; \genstateyArg{n}{\timeStartArg{n}} =
\begin{cases} \genstateArg{n-1}{\timeEndArg{n-1}} & n = 2,\ldots,\nslabs \\
\genstateIC & n=1, \end{cases} 
\end{split}
\end{equation}
%\begin{equation}\label{eq:obj_gen_slab}
%\begin{split}
%      &\genstateArgnt{n} =   \underset{\genstatey \in \RR{\romdim}}{\text{arg\,min}}\; \mathcal{J}^n(\basisspace \genstatey + \stateIntercept),\\ 
%      &\text{subject to }\; \genstateArg{n}{\timeStartArg{n}} =
%\begin{cases} \genstateArg{n-1}{\timeEndArg{n-1}} & n = 2,\ldots,\nslabs \\
%\genstateIC & n=1, \end{cases} 
%\end{split}
%\end{equation}
%
%Substituting the approximation~\eqref{eq:affine_trialspace_tclsrm} into the minimization problem~\eqref{eq:tclsrm}, the
%constrained minimization problem over the $n$th time slab can be re-written as,
%\begin{equation}\label{eq:obj_gen_slab} \genstate^n =
%\underset{\genstatey}{\text{arg min }} \mathcal{J}^n \bigg( \basisspace
%\genstatey + \stateIntercept \bigg) , \end{equation}
%%\begin{equation}\label{eq:obj_gen_slab} \genstate^n(t) =
%%\underset{\genstatey}{\text{argmin }} \mathcal{J}^n
%%\bigg(\decoder\big(\genstatey(\tau)\big) \bigg) , \end{equation}
%subject to the boundary conditions 
%\begin{equation}\label{eq:bcs}
%\genstate^n(\timeStartArg{n}) = \begin{cases} \genstate^{n-1}(\timeEndArg{n-1}
%) & n = 2,\ldots,\nslabs, \\ \basisspace^T(\stateFOMIC - \stateIntercept) &
%n=1, \end{cases} \end{equation}
where we have used $\elltwo$ projection of the initial conditions. 
%$\genstate^n(\timeStartArg{n}) = \genstate^{n-1}(\timeEndArg{n-1} )$. 
%In the above, $\genstate^n : [\timeStartArg{n} ,
%\timeEndArg{n} ] \rightarrow \RR{\romdim}$ are the generalized coordinates over the $n$th time slab.
At this point it is emphasized that~\eqref{eq:obj_gen_slab} corresponds to
an infinite dimensional minimization problem: the problem seeks to find the
\textit{function} $\genstate^n$ that minimizes the \textit{functional}
$\objectiveArg{n}$.

\subsubsection{Stationary conditions and the Euler--Lagrange equations}
Stationary conditions for~\eqref{eq:obj_gen_slab} can be obtained using the
calculus of variations. In particular, the calculus of variations can be used
to derive the Euler--Lagrange equations, the solutions of which are stationary 
points of the objective functional. For notational purposes, denote the
integrand of~\eqref{eq:obj} (written in terms of the generalized
coordinates) as, 
\begin{equation}\label{eq:integrand}
\begin{split}
 \minintegrandArg{n} & \vcentcolon
(\genstateyDiscreteArgnt{}, \genstateyDiscreteDotArgnt{},t) \mapsto \frac{1}{2} \big[
\basisspace \genstateyDiscreteDotArgnt{} - \velocity(\basisspace \genstateyDiscreteArgnt{}
+ \stateIntercept,t) \big]^T \stweightingMatArgt{n}{t} \big[
\basisspace \genstateyDiscreteDotArgnt{}  - \velocity(\basisspace \genstateyDiscreteArgnt{} +
\stateIntercept,t) \big], \\ & \vcentcolon \RR{\romdim} \times \RR{\romdim} \times [0,T]
 \rightarrow \RR{} .  
\end{split}
\end{equation}
%$$\minintegrand^n(\genstate^n,\dot{\genstate},t)= \frac{1}{2}
%\big[\frac{\partial \decoder}{\partial \genstate} \dot{\genstate} -
%\velocity(\decoder(\genstate)) \big]^T \stweightingMat(t) \big[\frac{\partial
%\decoder}{\partial \genstate} \dot{\genstate} -
%\velocity(\decoder(\genstate)) \big].$$
Using this notation, the Euler--Lagrange equations (see
Appendix~\ref{appendix:eulerlagrange} for the derivation) over the $n$th
time window are given by, 
\begin{equation}\label{eq:el1} 
\begin{split}
& \frac{\partial
\minintegrandArg{n}}{\partial
\genstateyDiscrete}(\genstateArg{n}{t},\genstateDotArg{n}{t},t) - \frac{d}{dt}
 \frac{\partial \minintegrandArg{n}}{\partial \genstateyDiscreteDotArgnt{}} 
(\genstateArg{n}{t},\genstateDotArg{n}{t} ,t) = \bz, \qquad t \in [\timeStartArg{n},\timeEndArg{n}],\\ 
&\genstate^n(\timeStartArg{n})  = \begin{cases} 
\genstate^{n-1}(\timeEndArg{n-1}) &
n=2,\ldots,\nslabs, \\ 
\basisspace^T(\stateFOMIC - \stateIntercept) & n=1,
\end{cases}\\ 
&\bigg[ \frac{\partial \minintegrandArg{n}}{\partial \genstateyDiscreteDotArgnt{}}(\genstateArg{n}{\timeEndArg{n}},\genstateDotArg{n}{\timeEndArg{n}},\timeEndArg{n})\bigg]^T =
\bz.\end{split} 
\end{equation}
Evaluating the terms in~\eqref{eq:el1} is an exercise in
vector calculus, and the step-by-step process is given in
Appendix~\ref{appendix:vector_calc}. The resulting system of equations is
given by the second-order differential equation,
%$$ \bigg[\basisspace^T \bigg[\frac{\partial \velocity}{\partial \state}
%\bigg]^T \stweightingMat  +  \frac{d}{dt} \bigg] \bigg(  \basisspace^T
%\stweightingMat \basisspace \dot{\genstate}   -  \basisspace^T
%\stweightingMat \velocity \bigg) = 0,$$
\EP{Working on this}
\begin{equation}\label{eq:clspg_2ord}
\begin{split} 
&\bigg[\basisspace^T \bigg[\frac{\partial
\velocity}{\partial \stateyDiscreteArgnt{}} (\basisspace \genstateArg{n}{t} +
\stateIntercept,t)\bigg]^T \stweightingMatArgt{n}{t} + \basisspace^T
\stweightingMatArgt{n}{t} \bigg] \bigg(  \basisspace
\genstateDDotArg{n}{t}    -  \velocityDot(\basisspace \genstateArg{n}{t} + \stateIntercept,t) \bigg) = \bz, \qquad t \in  [\timeStartArg{n},\timeEndArg{n}],
\\
&\bigg[\basisspace^T \bigg[\frac{\partial
\velocity}{\partial \stateyDiscreteArgnt{}} (\basisspace \genstateArg{n}{t} +
\stateIntercept,t)\bigg]^T \stweightingMatArgt{n}{t} + \basisspace^T
\stweightingMatArgt{n}{t} \frac{d}{dt} \bigg] \bigg(  \basisspace
\genstateDotArgnt{n}    -  \velocity^*(\basisspace \genstateArgnt{n} + \stateIntercept,t) \bigg) = \bz, \qquad t \in  [\timeStartArg{n},\timeEndArg{n}],
\\
%& \genstateArg{n}{ \timeStartArg{n}} = \genstateArg{n-1}{\timeEndArg{n-1}},
%\\
&\genstate^n( \timeStartArg{n})  = \begin{cases}
\genstate^{n-1}(\timeEndArg{n-1}) & n=2,\ldots,\nslabs, \\
\basisspace^T(\stateFOMIC - \stateIntercept) & n=1, \end{cases}\\ 
& 
\basisspace^T \stweightingMatArgt{n}{\timeEndArg{n}} \basisspace \genstateDotArg{n}{\timeEndArg{n}}  -
\basisspace^T \stweightingMatArgt{n}{\timeEndArg{n}}
\velocity(\basisspace \genstateArg{n}{\timeEndArg{n}} + \stateIntercept,\timeEndArg{n}) = \bz.  
\end{split}
\end{equation}
The second-order system~\eqref{eq:clspg_2ord} can be written equivalently as two first-order
equations by introducing an auxiliary ``\adjointStr", or ``costate", variable. %To do this, first
Defining the costate over the $n$th window as, 
\begin{equation}\label{eq:costate_def}
\begin{split}
\adjointArgnt{n} &: t \mapsto \genstateDotArg{n}{t}  -  \mass^{-1} \basisspace^T \stweightingMatArgt{n}{t}\velocity(\veloargsromn) ,\\
&: [\timeStartArg{n},\timeEndArg{n}] \rightarrow \RR{K} ,
\end{split}
\end{equation}
%\velocity(\veloargsromn)  
%\adjointArg{n}{t}
%\defeq 
%\genstateDotArg{n}{t}  -  \mass^{-1} \basisspace^T \stweightingMatArgt{n}{t}
%\velocity(\veloargsromn) , \end{align}
%\basisspace \frac{d \genstate}{dt}  -   \velocity =  \adjoint  , \qquad
%\genstate(t=0) = \genstate_0 \end{equation}
where
$\massArgnt{n} \equiv \basisspace^T \stweightingMatArgt{n}{t} \basisspace$ is the
mass matrix,
we can manipulate the system~\eqref{eq:clspg_2ord} (see
Appendix~\ref{appendix:vector_calc}) to obtain a coupled forward-backward
system for $t \in  [\timeStartArg{n},\timeEndArg{n}]$,
\begin{align}\label{eq:lspg_continuous} 
&\mass   \genstateDotArg{n}{t}  -  \basisspace^T
\stweightingMatArgt{n}{t} \velocity(\veloargsromn) =  \mass \adjointArg{n}{t} , \\
%\end{equation} \begin{equation}\label{eq:lspg_adjoint}
 &\mass  \adjointDotArg{n}{t}  + \basisspace^T \bigg[\frac{\partial
\velocity}{\partial \stateyDiscrete} (\basisspace \genstateArg{n}{t} +
\stateIntercept,t) \bigg]^T \stweightingMatArgt{n}{t} \basisspace
 \adjointArg{n}{t} = \nonumber \\ &-\basisspace^T \big[
\frac{\partial \velocity}{\partial \stateyDiscrete}(\basisspace \genstateArg{n}{t} +
\stateIntercept,t) \big]^T \stweightingMatArgt{n}{t} \bigg( \mathbf{I} -
\basisspace [\massArgnt{n}]^{-1} \basisspace^T \stweightingMatArgt{n}{t} \bigg)
 \bigg( \basisspace \genstateDotArg{n}{t} -
\velocity(\veloargsromn) \bigg), \label{eq:lspg_adjoint} \\ &
\genstateArg{n}{\timeStartArg{n}} = \begin{cases}
\genstate^{n-1}(\timeEndArg{n-1}) & n=2,\ldots,\nslabs, \label{eq:lspg_bcs1}\\
\basisspace^T(\stateFOMIC - \stateIntercept) & n=1, \end{cases}\\ &
\adjointArg{n}{\timeEndArg{n}} = \boldsymbol 0. \label{eq:lspg_bcs} 
\end{align} 
%where
%$\massArgnt{n}= \basisspace^T \stweightingMatArgt{n}{t} \basisspace$ is the
%mass matrix.
%\end{equation}
Equation~\eqref{eq:lspg_continuous} is equivalent to a Galerkin reduced-order
model forced by the costate variable, $\adjoint$.
Equation~\eqref{eq:lspg_adjoint} is typically referred to as the adjoint
equation, and corresponds to a linear equation that is forced by the residual.
It is worth noting that both~\eqref{eq:lspg_continuous}
and~\eqref{eq:lspg_adjoint} can be hyper-reduced through, e.g.,
collocation, (discrete) empirical interpolation, Gappy POD, etc.~\cite{everson_sirovich_gappy,eim,qdeim_drmac}. The
coupled system defined by~\eqref{eq:lspg_continuous}-\eqref{eq:lspg_bcs} can be interpreted as an ``optimally controlled"
ROM. The adjoint equation controls the forward model by enforcing the minimum
residual condition.
%Note that, $$ \basisspace^T \bigg[\frac{\partial \velocity}{\partial \state}
%\bigg]^T \basisspace \adjoint =  \basisspace^T \bigg[\frac{\partial
%\velocity}{\partial \state} \bigg]^T \stweightingMat \basisspace \adjoint .$$

A simplification is obtained in the basic case %with no
%hyper-reduction/weighting 
where $\stweightingMat = \mathbf{I}$.
The system becomes,
\begin{align*}\label{eq:lspg_continuous_simle} & \genstateDotArg{n}{t}  -
\basisspace^T  \velocity(\veloargsromn) =  \adjointArg{n}{t} , \\
%\end{equation} \begin{equation}\label{eq:lspg_adjoint}
 &\adjointDotArg{n}{t}   + \basisspace^T \bigg[\frac{\partial
\velocity}{\partial \stateyDiscrete}(\basisspace \genstateArg{n}{t} +
\stateIntercept,t)\bigg]^T \basisspace \adjointArg{n}{t} = \basisspace^T \bigg[
\frac{\partial \velocity}{\partial \stateyDiscrete} (\basisspace \genstateArg{n}{t} +
\stateIntercept,t) \bigg]^T \bigg( \mathbf{I} -   \basisspace \basisspace^T
\bigg)    \velocity(\veloargsromn) , \\ & \genstateArg{n}{\timeStartArg{n}} =
\begin{cases} \genstate^{n-1}(\timeEndArg{n-1}) & n=2,\ldots,\nslabs, \\
\basisspace^T(\stateFOMIC - \stateIntercept) & n=1, \end{cases}\\
&\adjointArg{n}{\timeEndArg{n}} = \boldsymbol 0 .  \end{align*}
\begin{remark}
Solutions to the Euler--Lagrange equations correspond to stationary points 
of the objective function~\eqref{eq:obj_gen_slab}. There are no guarantees, 
however, that a stationary point is a minimum. In general, the stationary 
points could be a maximum, a saddle point, a local minimum, etc.
\end{remark}

\subsubsection{Formulation as an optimal control problem of Lagrange type}\label{sec:optimal_control} The
Euler--Lagrange equations expose that~\eqref{eq:lspg_continuous}-\eqref{eq:lspg_bcs} can be alternatively formulated from as a Lagrange
problem from optimal control. To this end, consider the Galerkin ROM over the $n$th window, 
$$ \basisspace^T \stweightingMatArgt{n}{t} \basisspace
 \genstateGalerkinDotArg{n}{t} - \basisspace^T \stweightingMatArgt{n}{t}
\velocity(\basisspace \genstateGalerkinArg{n}{t} + \stateIntercept,t) = \boldsymbol 0.$$
We introduce now the controller, $\controllerArgnt{n} :  [\timeStartArg{n},\timeEndArg{n}] \rightarrow
\RR{\romdim}$,
and state that a residual minimizing controller exists such that \methodAcronym\ can be written as 
\begin{equation}\label{eq:controlled_rom}
 \basisspace^T
\stweightingMatArgt{n}{t} \basisspace \genstateDotArg{n}{t}  - \basisspace^T
\stweightingMatArgt{n}{t}\velocity(\basisspace \genstateArg{n}{t} +
\stateIntercept,t) = \controllerArg{n}{t}. 
 \end{equation}
%where $\mass \defeq \big[ \basisspace^T \stweightingMat \basisspace]^{-1} \in
%\RR{\romdim \times \romdim}$ is the \textit{mass matrix}. 
We now demonstrate how to find this controller.
Before doing so, it is noted that~\eqref{eq:controlled_rom} displays commonalities with \textit{subgrid-scale}
methods~\cite{iliescu_pod_eddyviscosity,iliescu_vms_pod_ns,iliescu_ciazzo_residual_rom,parish_apg,wentland_apg,Wang:269133,San2018},
in where an additional term is added to the reduced-order model so-as to
account for the truncated dynamics. 

We start by defining a \textit{Lagrangian} that measures the residual norm as a function of the state and controller, 
\begin{equation}\label{eq:obj_controller}
\begin{split}
%\objectiveControlArg{n}(\stateArg{}{t}) = \frac{1}{2}
%\int_{\timeStartArg{n}}^{\timeEndArg{n}} \big[ \basisspace
%\dot{\genstate}(\tau) - \velocity(\stateArg{}{\tau} ) \big]^T
%\stweightingMat(\tau) \big[ \dot{\state}(\tau) - \velocity(\stateArg{}{\tau})
%\big] d \tau, \objectiveControlArg{n}(\genstateArg{}{t},\controllerArg{}{t})
%= \\ \frac{1}{2} \bigg[ \basisspace \bigg(  \mass^{-1}\basisspace^T
%\stweightingMat\velocity(\basisspace \genstate) + \mass^{-1}\controller
%\bigg) - \velocity(\basisspace \genstate ) \bigg]^T \stweightingMat(t) \bigg[
%\basisspace \bigg(  \mass^{-1}\basisspace^T
%\stweightingMat\velocity(\basisspace \genstate) + \mass^{-1}\controller
%\bigg) - \velocity( \basisspace  \genstate ) \bigg] , \\
 \objectiveControlArg{} &:  (\genstateyDiscreteArgnt{},\controllerDiscreteDumArgnt{},t)
\mapsto \frac{1}{2} \bigg[ \basisspace \bigg(  \mass^{-1}\basisspace^T
\stweightingMatArgt{n}{t}  \velocity(\basisspace \genstateyDiscreteArgnt{} +
\stateIntercept,t) + \mass^{-1}\controllerDiscreteDumArgnt{} \bigg) -
\velocity(\basisspace \genstateyDiscreteArgnt{} + \stateIntercept,t) \bigg]^T
\stweightingMatArgt{n}{t}  \\ & \hspace{1.5 in}\bigg[ \basisspace \bigg(
\mass^{-1}\basisspace^T \stweightingMatArgt{n}{t}\velocity(\basisspace
\genstateyDiscreteArgnt{} + \stateIntercept,t) + \mass^{-1}\controllerDiscreteDumArgnt{}
\bigg) - \velocity( \basisspace  \genstateyDiscreteArgnt{} + \stateIntercept,t ) \bigg]
, \nonumber \\ & : \RR{\romdim} \times \RR{\romdim} \times [0,T] \rightarrow \RR{},
 %\int_{\timeStartArg{n}}^{\timeEndArg{n}}
 %\objectiveControlArg{n}(\genstateArg{}{\tau},\controllerArg{}{\tau}) d\tau =
 %\\ \frac{1}{2} \int_{\timeStartArg{n}}^{\timeEndArg{n}} \bigg[ \basisspace
 %\bigg(  \mass^{-1}\basisspace^T \stweightingMat\velocity(\basisspace
 %\genstate) + \mass^{-1}\controller \bigg) - \velocity(\basisspace \genstate
 %) \bigg]^T \stweightingMat(\tau) \bigg[ \basisspace \bigg(
 %\mass^{-1}\basisspace^T \stweightingMat\velocity(\basisspace \genstate) +
 %\mass^{-1}\controller \bigg) - \velocity( \basisspace  \genstate ) \bigg] d
 %\tau,
\end{split}
\end{equation}
where it is noted that, by~\eqref{eq:controlled_rom}, $
[\massArgnt{n}]^{-1}\basisspace^T \stweightingMatArg{n} \velocity(\basisspace \genstateArg{n}{t} + \stateIntercept ,t) +
\mass^{-1}\controllerArg{n}{t}  = \genstateDotArg{n}{t}.$ 
Hence~\eqref{eq:obj_controller} is a measure of the same residual as in~\eqref{eq:obj}.  The \methodAcronym\ method can be formulated as a minimization problem 
that computes the controller $\controllerArgnt{n}$, for $n=1,\ldots,\nslabs$, as the solution to the minimization problem, 
\begin{equation}\label{eq:tclspg_oc1a} 
\begin{split}
&\underset{\controllerDumArgnt{n} }{\text{minimize } } 
\int_{\timeStartArg{n}}^{\timeEndArg{n}}
\objectiveControlArg{}(\genstateArg{n}{t},\controllerDumArg{n}{t},t)dt, 
 \\
&\text{subject to } \;  \left\{\begin{array}{l} 
 \basisspace^T \stweightingMatArgt{n}{t}
\basisspace  \genstateDotArg{n}{t}  - \basisspace^T
\stweightingMatArgt{n}{t} \velocity(\basisspace \genstateArg{n}{t} +
\stateIntercept,t) =\controllerDumArg{n}{t}, \\
 \genstateArg{n}{\timeStartArg{n}} =
\begin{cases} \genstate^{n-1}(\timeEndArg{n-1} ) & n = 2,\ldots,\nslabs,
 \\ \basisspace^T(\stateFOMIC - \stateIntercept) & n=1. \end{cases} \end{array} \right.
% \genstateArg{n-1}{\timeEndArg{n-1}}.
\end{split}
\end{equation}
%\begin{equation}\label{eq:tclspg_oc1a} 
%\begin{split}
%&\controllerArgnt{n} =
%\underset{\controllerDumArgnt{} }{\text{arg\,min } } 
%\int_{\timeStartArg{n}}^{\timeEndArg{n}}
%\objectiveControlArg{}(\genstateArg{n}{t},\controllerDumArg{}{t},t)dt, 
% \\
%&\text{subject to } \;  \left\{\begin{array}{l} 
% \basisspace^T \stweightingMatArgt{n}{t}
%\basisspace \frac{d}{dt} \genstateArg{n}{t}  - \basisspace^T
%\stweightingMatArgt{n}{t} \velocity(\basisspace \genstateArg{n}{t} +
%\stateIntercept,t) =\controllerArg{n}{t}, \\
% \genstateArg{n}{\timeStartArg{n}} =
%\begin{cases} \genstate^{n-1}(\timeEndArg{n-1} ) & n = 2,\ldots,\nslabs,
% \\ \basisspace^T(\stateFOMIC - \stateIntercept) & n=1. \end{cases} \end{array} \right.
%% \genstateArg{n-1}{\timeEndArg{n-1}}.
%\end{split}
%\end{equation}
%Noting that
%$$\basisspace^T \stweightingMat \basisspace \dot{\genstate} = \basisspace^T \controller + \basisspace^T \stweightingMat \velocity(\basisspace \genstate).$$
%$$\basisspace \dot{\genstate} =  \basisspace \controller + \basisspace \basisspace^T \velocity(\basisspace \genstate).$$
%Then
%\begin{equation}\label{eq:obj}
%\objectiveControlArg{n}(\stateArg{}{t}) = \frac{1}{2} \int_{\timeStartArg{n}}^{\timeEndArg{n}} \big[\basisspace \controller + (\basisspace \basisspace^T  - \mathbf{I} )\velocity(\basisspace \genstate)   ) \big]^T \stweightingMat(\tau) \big[\basisspace \controller + (\basisspace \basisspace^T - \mathbf{I}) \velocity(\stateArg{}{\tau})  \big] d \tau,
%\end{equation}
The solution to the system~\eqref{eq:tclspg_oc1a} is equivalent of that defined by~\eqref{eq:obj_gen_slab}. 
This can be demonstrated via the \textit{Pontryagin Maximum Principle} (PMP)~\cite{optimal_control_book}. To this end, we introduce the Lagrange multiplier (costate) 
$\adjointOCArgnt{n}: [\timeStartArg{n},\timeEndArg{n}] \mapsto \RR{\romdim}$ and define the Hamiltonian, 
\begin{equation}\label{eq:hamiltonian}
\begin{split}
\hamiltonian \; &: \;  (\genstateyDiscreteArgnt{},\adjointDiscreteDumArgnt{},\controllerDiscreteDumArgnt{},t) \mapsto 
 \adjointDiscreteDumArgnt{T} \bigg[  \mass^{-1}\basisspace^T \stweightingMatArgt{n}{t}\velocity(\basisspace \genstateyDiscreteArgnt{} + \stateIntercept,t) + \mass^{-1}\controllerDiscreteDumArgnt{} \bigg] +  \objectiveControlArg{}(\genstateyDiscreteArgnt{},\controllerDiscreteDumArgnt{},t) \\
&: \; \RR{\romdim} \times \RR{\romdim} \times \RR{\romdim} \times [0,T] \rightarrow \RR{}.
\end{split}
\end{equation} 
%The Pontryagin Maximum Principle states that that 
%These stationary points can be obtained with the \textit{Pontryagin Maximum Principle} (PMP)~\cite{PMP}. In o do so, define the Hamiltonian,
%$$\hamiltonian^n(\genstate^n,\adjoint^n,\controller^n) = \adjoint^T \bigg[  \mass^{-1}\basisspace^T \stweightingMat\velocity(\basisspace \genstate^n) + \mass^{-1}\controller^n \bigg] +  \objectiveControlArg{n}(\genstate^n,\controller^n)$$
The Pontryagin Maximum Principle states that, for a trajectory to be an extremal of~\eqref{eq:tclspg_oc1a} (with the dynamics constraints), it must satisfy the 
following conditions over the $n$th window,
\begin{align*}%\label{eq:hamiltonian_sys}
&\genstateDotArg{n}{t} = \frac{\partial \hamiltonianArg{}}{\partial \adjointDiscreteDumArgnt{}{}}(\genstateArg{n}{t},\adjointOCArg{n}{t},\controllerArg{n}{t},t),\\% \qquad \genstate(t_0(n)) = \genstate^{n-1}(t_1(n-1))\\
&\adjointOCDotArg{n}{t}= - \frac{\partial \hamiltonianArg{}}{\partial \genstateyDiscreteArgnt{}{}}(\genstateArg{n}{t},\adjointOCArg{n}{t},\controllerArg{n}{t},t),\\% \qquad \adjoint^n(t_1(n)) = 0\\
&\frac{\partial \hamiltonianArg{}}{\partial \controllerDiscreteDumArgnt{}{}} (\genstateArg{n}{t},\adjointOCArg{n}{t},\controllerArg{n}{t},t) = \boldsymbol 0, \\
& \genstateArg{n}{\timeStartArg{n}} =
\begin{cases} \genstate^{n-1}(\timeEndArg{n-1} ) & n = 2,\ldots,\nslabs,
 \\ \genstateIC & n=1,\end{cases} \\
&\adjointOCArg{n}{\timeEndArg{n}}= \bz.
\end{align*}
Evaluation of the required gradients (Appendix~\ref{appendix:optimal_control}) yields the system to be solved over the $n$th window for $t \in [\timeStartArg{n},\timeEndArg{n}]$,
\begin{align}\label{eq:sys_oc1}
&\mass  \genstateDotArg{n}{t}  -  \basisspace^T \stweightingMatArgt{n}{t} \velocity(\veloargsromn) =  \controllerArg{n}{t} , \\
%\end{equation}
%\begin{equation}\label{eq:lspg_adjoint}
 & \adjointOCDotArg{n}{t}  + \basisspace^T \bigg[\frac{\partial \velocity}{\partial \stateyDiscrete} (\basisspace \genstateArg{n}{t} + \stateIntercept,t) \bigg]^T \stweightingMatArgt{n}{t} \basisspace [\massArgt{n}{t}]^{-1} \adjointOCArg{n}{t} = \nonumber \\
 & \qquad \basisspace^T \big[ \frac{\partial \velocity}{\partial \stateyDiscrete}(\basisspace \genstateArg{n}{t} + \stateIntercept,t) \big]^T \stweightingMatArgt{n}{t} \bigg( \mathbf{I} -   \basisspace [\massArgt{n}{t}]^{-1} \basisspace^T  \stweightingMatArgt{n}{t} \bigg)  \bigg( \basisspace \genstateDotArg{n}{t}  -   \velocity(\veloargsromn) \bigg) \label{eq:sys_oc2} \\
&\controllerArg{n}{t} = -\adjointOCArg{n}{t} \label{eq:sys_oc3}, \\
& \genstateArg{n}{\timeStartArg{n}} = 
\begin{cases}
 \genstate^{n-1}(\timeEndArg{n-1}) & n=2,\ldots,\nslabs, \\
\genstateIC & n=1,  
\end{cases} \label{eq:sys_oc4} \\
& \adjointOCArg{n}{\timeEndArg{n}} = \boldsymbol 0 \label{eq:sys_oc5}.
\end{align}
%where $\massArgt{n}{t} = \basisspace^T \stweightingMatArgt{n}{t} \basisspace$ is the mass matrix.
%\begin{align}
%&\frac{d \genstate}{dt}  - \mass^{-1}\basisspace^T \stweightingMat\velocity(\basisspace \genstate) = \mass^{-1}\controller, \label{eq:sys_oc1} \\
%& \frac{d \adjoint}{dt} + \basisspace^T \bigg[\frac{\partial \velocity}{\partial \state} \bigg]^T \stweightingMat \basisspace \mass^{-1} \adjoint^n = 
%  \basisspace^T \bigg[\frac{\partial \velocity}{\partial \state} \bigg]^T \stweightingMat \bigg( \mathbf{I} - \basisspace \mass^{-1} \basisspace^T \bigg) \stweightingMat  \bigg( \basisspace \dot{\genstate}^n   -   \velocity(\veloargsromn) \bigg) \bigg] , \qquad \adjoint^n(t_1(n)) = 0, \label{eq:sys_oc2} \\
%&\controller = \adjoint.
%\end{align}
%\begin{align}%\label{eq:lspg_continuous_oc}
%&\basisspace^T \stweightingMatArg{n} \basisspace \frac{d \genstate^n}{dt}  -  \basisspace^T \stweightingMatArg{n} \velocity(\veloargsromn) =  \controller^n , \qquad \genstate(t_0(n)) = \genstate^{n-1}(t_1(n-1)), \label{eq:sys_oc1} \\
%%\end{equation}
%%\begin{equation}\label{eq:lspg_adjoint}
% &\frac{d}{dt} \adjoint^n  + \basisspace^T \bigg[\frac{\partial \velocity}{\partial \state} \bigg]^T \stweightingMatArg{n} \basisspace \mass^{-1} \adjoint^n = -\bigg[\basisspace^T \frac{\partial \velocity}{\partial \state^n}^T \stweightingMatArg{n} \bigg( \mathbf{I} -   \basisspace \mass^{-1} \basisspace^T \bigg)  \stweightingMatArg{n} \bigg( \basisspace \dot{\genstate}^n   -   \velocity(\veloargsromn) \bigg) \bigg] , \qquad \adjoint^n(t_1(n)) = 0, \label{eq:sys_oc2}\\
%&\controller^n = \adjoint^n \label{eq:sys_oc3}.
%\end{align}           
The system defined by~\eqref{eq:sys_oc1}-\eqref{eq:sys_oc5} is seen to be equivalent to that defined by~\eqref{eq:lspg_continuous}-\eqref{eq:lspg_bcs} (perform 
the change of variables $\controllerArgnt{n} = \mass \adjointArgnt{n}$ in~\eqref{eq:sys_oc1} and $\adjointOCArgnt{n} = -\mass \adjointArgnt{n}$ in~\eqref{eq:sys_oc2}). 
Thus, \methodAcronym\ with \spatialAcronym\ trial subspaces be formulated as an optimal control problem: \methodAcronym\ finds a controller that modifies the Galerkin ROM such that the residual minimization objective is achieved. In addition to commonalities with optimal control problems, this second formulation displays commonalities with subgrid-scale modeling techniques, in where an additional term is added to the reduced-order model to account for the impact of the truncated modes. \methodAcronym\ can thus be interpreted as a subgrid-scale modeling technique, in where a residual minimizing subgrid-scale model is constructed.

\begin{remark}
The Euler--Lagrange equations comprise a Hamiltonian system. This imbues \methodAcronym\ with certain properties; e.g., for autonomous systems the Hamiltonian~\eqref{eq:hamiltonian} is conserved. 
\end{remark} 


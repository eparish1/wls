%\newcommand{\methodAcronym}{WLS}
%\newcommand{\methodAcronymROM}{WLS-ROM}
%\newcommand{\methodAcronymROMs}{WLS-ROMs}
%
%\newcommand{\methodName}{Windowed Least-Squares}
\newcommand{\directMethodAcronym}{direct-WLS}
\newcommand{\methodAcronym}{WLS}
\newcommand{\methodAcronymROM}{WLS-ROM}
\newcommand{\methodAcronymROMs}{WLS-ROMs}

\newcommand{\methodName}{Windowed Least-Squares}
\newcommand{\methodNameLower}{windowed least-squares}
\newcommand{\EP}[1]{{\color{red}EP: #1}}
%======= Shaping for matrix
%\newcommand{\matshapea}{\underbrace{[ \circ \circ \circ \circ \circ ]}_{1 \times N}}
%\newcommand{\matshapeb}{\begin{matrix} \circ &   \\ \circ & \circ    \end{matrix}}
%\newcommand{\matshapebf}{\begin{matrix} \circ & \circ  \\ \circ & \circ    \end{matrix}}
%\newcommand{\matshapec}{\circ}
\newcommand{\matshapea}{
\tikz[baseline=0.0ex]\draw (0,0) rectangle (15pt,-25pt);}
\newcommand{\matshapeb}{
\tikz[baseline=0.0ex]\draw (0,0) rectangle (10pt,-25pt);}


%\newcommand{\matshapeb}{\begin{bmatrix} &   \end{bmatrix}}
%\newcommand{\matshapeb}{\tikz[baseline=0.0ex]\draw (0,0) triangle (30pt,30pt,30pt);}
%\newcommand{\matshapeb}{
%%\begin{tikzpicture}
%\tikz[baseline=0.0ex]\draw (0,0)%{$A$}
%  -- (20pt,0pt) %{$C$}
%  -- (0pt,20pt) %{$B$}
%  -- cycle;}
%\end{tikzpicture}}
\newcommand{\matshapebf}{
\tikz[baseline=0pt]\draw (0pt,0pt) rectangle (20pt,20pt);}
%\newcommand{\matshapec}{\rule[0pt]{0.5pt}{20pt}}
\newcommand{\matshapec}{
\tikz[baseline=0.0ex]\draw (0,0pt) rectangle (2pt,20pt);}
%====================
\newcommand{\stdim}{K_{\text{ST}}}
\newcommand{\stdimArg}[1]{K_{\text{ST}}^{#1}}
\newcommand{\quadWeightsScalarArg}[3]{\alpha^{#1,#2}_{#3}}
\newcommand{\quadWeightsLMSScalarArg}[2]{\gamma^{#1,#2}}
\newcommand{\timeDumArg}{t}
\newcommand{\timeDummy}{\tau}
\newcommand{\fbsmGrowth}{\psi_1}
\newcommand{\fbsmDecay}{\psi_2}
\newcommand{\projector}{\mathbb{P}}
\newcommand{\projectorArg}[1]{\projector^{#1}}
\newcommand{\veloargsromy}{\basisspace \genstateyDiscreteArgnt{} + \stateIntercept,t}
\newcommand{\quadWeightsVecArg}[1]{\boldsymbol \alpha^{#1}}
\newcommand{\minIntegrandFull}{\frac{1}{2} \big[\frac{\partial \decoder}{\partial \genstate} \dot{\genstate} - \velocity(\decoder(\genstate)) \big]^T \stweightingMat(t) \big[\frac{\partial \decoder}{\partial \genstate} \dot{\genstate} - \velocity(\decoder(\genstate)) \big]}

\newcommand{\adjointStr}{adjoint}
%\newcommand{\stweightingMatCollocArg}[2]{\stweightingMat^{#1,#2}}
%\newcommand{\stweightingMatCollocArgt}[2]{\stweightingMat^{#1}(#2)}
\newcommand{\stweightingMatCollocArgt}[2]{\stweightingMat}
\newcommand{\stweightingMatCollocArg}[2]{\overline{\stweightingMat}}
\newcommand{\stweightingMat}{ \mathbf{A}}

\newcommand{\stweightingMatOne}{ \mathbf{W}}
\newcommand{\stweightingMatOneArg}[1]{\stweightingMatOne^{#1}}
\newcommand{\stweightingMatOneTArg}[1]{[\stweightingMatOne^{#1}]^T}

%\newcommand{\stweightingMatArg}[1]{\stweightingMat^{#1}}
%\newcommand{\stweightingMatArgt}[2]{\stweightingMat^{#1}(#2)}
\newcommand{\stweightingMatArg}[1]{\stweightingMat^{#1}}
\newcommand{\stweightingMatArgt}[2]{\stweightingMat^{#1}}
\newcommand{\DeltaSlabArg}[1]{\Delta T^{#1}}
\newcommand{\lagrangeBasis}{\boldsymbol \ell}
\newcommand{\lagrangeBasisArg}[2]{\boldsymbol \ell^{#1,#2}}
\newcommand{\lagrangeBasisScalar}{ \ell}
\newcommand{\lagrangeBasisScalarArg}[3]{\lagrangeBasisScalar^{#1,#2}_{#3}}

\newcommand{\collocBasis}{\boldsymbol \lagrangeBasis}
\newcommand{\collocBasisArg}[2]{\collocBasis^{#1,#2}}
\newcommand{\collocBasisDot}{\dot{\collocBasis}}
\newcommand{\collocBasisDotArg}[2]{\collocBasisDot^{#1,#2}}
\newcommand{\collocBasisScalar}{p}
\newcommand{\collocBasisScalarArg}[4]{p^{#1,#2}_{#3}({#4})}
\newcommand{\jacobianSlab}{\boldsymbol J}
\newcommand{\jacobianSlabArg}[1]{\jacobianSlab^{#1}}
\newcommand{\hamiltonian}{\mathcal{H}}
\newcommand{\hamiltonianArg}[1]{\hamiltonian^{#1}}
\newcommand{\adjoint}{\hat{\boldsymbol \lambda}}

\newcommand{\adjointDot}{\dot{\adjoint}}
\newcommand{\adjointDotArg}[2]{\adjointDot^{#1}(#2)}

\newcommand{\adjointArg}[2]{\adjoint^{#1}(#2)}
\newcommand{\adjointArgnt}[1]{\adjoint^{#1}}

\newcommand{\adjointOCDot}{\dot{\hat{\boldsymbol \nu}}}
\newcommand{\adjointOCDotArg}[2]{\adjointOCDot^{#1}(#2)}

\newcommand{\adjointOC}{\hat{\boldsymbol \nu}}
\newcommand{\adjointOCArg}[2]{\adjointOC^{#1}(#2)}
\newcommand{\adjointOCArgnt}[1]{\adjointOC^{#1}}


\newcommand{\adjointDum}{\hat{\boldsymbol \mu}}
\newcommand{\adjointDumArgt}[2]{\adjointDum^{#1}(#2)}
\newcommand{\adjointDumArgnt}[1]{\adjointDum^{#1}}
\newcommand{\adjointDiscreteDum}{\hat{\pmb \mu}}
\newcommand{\adjointDiscreteDumArgt}[2]{\adjointDiscreteDum^{#1}(#2)}
\newcommand{\adjointDiscreteDumArgnt}[1]{\adjointDiscreteDum^{#1}}

\newcommand{\Range}[1]{\mathrm{Ran}(#1)}

\newcommand{\lipshitz}{\kappa}
\newcommand{\lipshitzi}{\alpha}
\newcommand{\lipshitziArg}[1]{\alpha^{#1}}

\newcommand{\lipshitzv}{\kappa_{\velocity}}
\newcommand{\constAInt}{\eta}
\newcommand{\stateFOMSol}{\stateFOM)}
\newcommand{\stateFOMDot}{\dot{\stateFOM}}
\newcommand{\stateFOMDotArg}[1]{\dot{\stateFOM}(#1)}

\newcommand{\stateFOMSolArg}[1]{\stateFOM^{#1}}
\newcommand{\stateFOMSolArgt}[2]{\stateFOM^{#1}(#2)}
\newcommand{\stateFOMProjSol}{\tilde{\state}_{\ell^2}}
\newcommand{\stateFOMProjSolArg}[1]{\stateFOMProjSol^{#1}}
\newcommand{\stateFOMProjSolArgt}[2]{\stateFOMProjSol^{#1}(#2)} 

\newcommand{\intSlabArg}[1]{\int_{\timeStartArg{#1}}^{\timeEndArg{#1}} }
\newcommand{\genstateROMSol}{\genstate}
\newcommand{\genstateROMSolArg}[1]{\genstateROMSol^{#1}}
\newcommand{\genstateROMSolArgt}[2]{\genstateROMSol^{#1}(#2)}
\newcommand{\genstateROMStarSol}{\genstate_{*}}
\newcommand{\genstateROMStarSolArg}[1]{\genstateROMStarSol^{#1}}
\newcommand{\genstateROMStarSolArgt}[2]{\genstateROMStarSol^{#1}(#2)}


%\newcommand{\approxstateIC}{\approxstateDiscrete_0}
\newcommand{\approxstateIC}{\basisspace \basisspace^T (\stateFOMIC - \stateIntercept) + \stateIntercept}

\newcommand{\stateROMSol}{\tilde{\state}}
\newcommand{\stateROMSolArg}[1]{\stateROMSol^{#1}}
\newcommand{\stateROMSolArgt}[2]{\stateROMSol^{#1}(#2)}
\newcommand{\stateROMStarSol}{\tilde{\state}_{*}}
\newcommand{\stateROMStarSolArg}[1]{\stateROMStarSol^{#1}}
\newcommand{\stateROMStarSolArgt}[2]{\stateROMStarSol^{#1}(#2)}
\newcommand{\errorROMStarSolArgt}[2]{\error^{#1}_{R^*}(#2)}
\newcommand{\errorConstantArg}[1]{\theta^{#1}}
\newcommand{\adjointROMStarSol}{\adjoint_*}
\newcommand{\adjointROMStarSolArg}[1]{\adjointROMStarSol^{#1}}

\newcommand{\adjointROMSol}{\adjoint}
\newcommand{\adjointROMSolArg}[1]{\adjoint^{#1}}
\newcommand{\adjointROMStarSolArgt}[2]{\adjointROMStarSol^{#1}(#2)}
\newcommand{\adjointROMSolArgt}[2]{\adjointROMSol^{#1}(#2)}

\newcommand{\errorArg}[1]{\error^{#1}}
\newcommand{\errorArgt}[2]{\error^{#1}(#2)}
\newcommand{\objectiveArg}[1]{\objective^{#1}}
\newcommand{\objectiveDiscrete}{J}
\newcommand{\objectiveArgLMS}[1]{\objectiveDiscrete_{}^{#1}}
\newcommand{\objectiveArgC}[1]{\objectiveDiscrete_{C}^{#1}}

\newcommand{\objectiveDiscreteST}{J_{\text{ST}}}
\newcommand{\objectiveDiscreteSTArg}[1]{\objectiveDiscreteST^{#1}}


%\newcommand{\objectiveDiscrete}{\overline{\objective}}
\newcommand{\objectiveDiscreteArg}[1]{\objectiveDiscrete^{#1}}

\newcommand{\objective}{\mathcal{J}}
\newcommand{\controller}{\hat{\boldsymbol u}}
\newcommand{\controllerArg}[2]{\controller^{#1}(#2)}
\newcommand{\controllerArgnt}[1]{{\controller^{#1}}}
\newcommand{\controllerStarArgnt}[1]{{\controller^{#1^*}}}

\newcommand{\objectiveControl}{\mathcal{L}}
\newcommand{\objectiveControlArg}[1]{\objectiveControl^{#1}}
\newcommand{\controllerDum}{\hat{\boldsymbol v}}
\newcommand{\controllerDumArg}[2]{\controllerDum^{#1}(#2)}
\newcommand{\controllerDumArgnt}[1]{\controllerDum^{#1}}
\newcommand{\controllerDiscreteDum}{\hat{\mathbf{v}}}
\newcommand{\controllerDiscreteDumArg}[2]{\controllerDiscreteDum^{#1}(#2)}
\newcommand{\controllerDiscreteDumArgnt}[1]{\controllerDiscreteDum^{#1}}

\newcommand{\collocModes}{\boldsymbol x}
\newcommand{\collocMat}{\mathbf{X}_\text{C}}
\newcommand{\collocMatArg}[2]{\collocMat^{#1,#2}}
\newcommand{\collocModesArg}[1]{\collocModes^{#1}}
\newcommand{\genCollocModes}{\overline{\collocModes}}
\newcommand{\genCollocModesArg}[1]{\overline{\collocModes}^{#1}}

\newcommand{\spatialIC}{\basisspaceTArg{n}(\basisspaceArg{n-1} \genstate^{n-1}(\timeEndArg{n-1} )  + \stateInterceptArg{n-1} - \stateInterceptArg{n} )}

\newcommand{\defeq}{\vcentcolon=}
\newcommand{\mass}{\mathbf{M}}
\newcommand{\massArg}[1]{\mass^{#1}}

%\newcommand{\massArgnt}[1]{\mass^{#1}}
\newcommand{\massArgnt}[1]{\mass}
\newcommand{\massArgt}[2]{\mass}
%\newcommand{\genstateyDotArg}[2]{\dot{\genstatey}^{#1}(#2)}
%\newcommand{\genstateyDot}{\dot{\genstatey}}
\newcommand{\genstateyDot}{\hat{\boldsymbol v}}
\newcommand{\genstateyDotArg}[2]{\genstateyDot^{#1}(#2)}


%\newcommand{\genstateyDiscreteDot}{\dot{\genstateyDiscrete}}
\newcommand{\genstateyDiscreteDot}{\hat{\mathbf{v}}}
%\newcommand{\genstateyDiscreteDotArgnt}[1]{\dot{\genstateyDiscrete}^{#1}}
\newcommand{\genstateyDiscreteDotArgnt}[1]{\genstateyDiscreteDot^{#1}}

\newcommand{\genstateDot}{\dot{\genstate}}
\newcommand{\stateDot}{\dot{\state}}
\newcommand{\stateyDot}{{\boldsymbol v}}

\newcommand{\statezDot}{\dot{\statez}}
\newcommand{\statez}{\boldsymbol z}
\newcommand{\statezDotBar}{\dot{\statezBar}}
\newcommand{\statezBar}{\overline{\statez}}

\newcommand{\variation}{\boldsymbol \eta}
\newcommand{\variationArgn}[1]{\boldsymbol \eta^{#1}}
\newcommand{\variationArgntt}[2]{\boldsymbol \eta^{#1}(#2)}

\newcommand{\variationDot}{\dot{\boldsymbol \eta}}

\newcommand{\variationArg}[1]{\boldsymbol \eta(#1)}
\newcommand{\variationArgnt}{\boldsymbol \eta}

\newcommand{\genstateyDotArgnt}[1]{\dot{\genstatey}^{#1}}

\newcommand{\genstateDDotArg}[2]{\ddot{\genstate}^{#1}(#2)}
\newcommand{\velocityDot}{\dot{\velocity}}

\newcommand{\genstateDotArg}[2]{\dot{\genstate}^{#1}(#2)}
\newcommand{\genstateDotArgnt}[1]{\dot{\genstate}^{#1}}


\newcommand{\genstateDiscrete}{\hat{\mathbf{x}}}
\newcommand{\genstateDiscreteArg}[1]{\genstateDiscrete^{#1}}
\newcommand{\genstateGuessDiscrete}{\hat{\mathbf{x}}}
\newcommand{\genstateGuessDiscreteArg}[2]{\genstateGuessDiscrete^{#1}_{#2}}

\newcommand{\genstateArg}[2]{\genstate^{#1}(#2)}
\newcommand{\genstateArgnt}[1]{\genstate^{#1}}

\newcommand{\genstateGalerkinDot}{\dot{\genstate}_{\text{G}}}
\newcommand{\genstateGalerkinDotArg}[2]{\genstateGalerkinDot^{#1}(#2)}

\newcommand{\genstateGalerkin}{\genstate_{\text{G}}}
\newcommand{\genstateGalerkinArg}[2]{\genstateGalerkin^{#1}(#2)}
\newcommand{\genstateGalerkinArgnt}[1]{\genstateGalerkin^{#1}}

\newcommand{\genstateLSPG}{\genstate_{\text{L}}}
\newcommand{\genstateLSPGArg}[2]{\genstateGalerkin^{#1}(#2)}
\newcommand{\genstateLSPGArgnt}[1]{\genstateGalerkin^{#1}}



\newcommand{\genstateyArg}[2]{\genstatey^{#1}(#2)}
\newcommand{\genstateyArgnt}[1]{\genstatey^{#1}}

\newcommand{\genstateyDiscreteArg}[1]{\genstateyDiscrete^{#1}}
\newcommand{\genstateyDiscreteArgnt}[1]{\genstateyDiscrete^{#1}}

\newcommand{\stateyArgnt}[1]{\statey^{#1}}
\newcommand{\stateyDiscreteArgnt}[1]{\stateyDiscrete^{#1}}
\newcommand{\stateyDiscreteArg}[1]{\stateyDiscrete^{#1}}

\newcommand{\ncolloc}{n_{\text{C}}}
\newcommand{\ncollocArg}[2]{\ncolloc^{#1,#2}}

\newcommand{\ncollocST}{N_{\text{ST}}}
\newcommand{\ncollocSTArg}[1]{\ncollocST^{#1}}
\newcommand{\collocPointST}{\chi_{\text{ST}}}
\newcommand{\collocPointSTArg}[2]{\collocPointST^{#1,#2}}


\newcommand{\ncollocEnforce}{\ncolloc + 1}
\newcommand{\collocOrder}{p}
\newcommand{\collocOrderArg}[2]{p^{#1,#2}}

\newcommand{\minintegrandArg}[1]{\minintegrand^{#1}}
%\newcommand{\minintegrandArg}[1]{\minintegrand}
\newcommand{\minintegrandCollocArg}[1]{\minintegrandColloc^{#1}}
\newcommand{\dIdV}{\minintegrand_{ \genstateyDiscreteDot}}
\newcommand{\dIdVDot}{\dot{\minintegrand}_{ \genstateyDiscreteDot}}
\newcommand{\dIdY}{\minintegrand_{ \genstateyDiscrete}}


\newcommand{\dIdVArg}[1]{\dIdV^{#1}}
\newcommand{\dIdYArg}[1]{\dIdY^{#1}}
\newcommand{\dIdVDotArg}[1]{\dIdVDot^{#1}}

\newcommand{\minintegrandDiscrete}{\minintegrand_c}
\newcommand{\minintegrandDiscreteArg}[1]{\minintegrand_c^{#1}}


\newcommand{\lagrange}{\ell}
\newcommand{\lagrangecoef}{c}


\newcommand{\collocPoint}{\zeta}
\newcommand{\collocPointArg}[3]{\zeta^{#1,#2}_{#3}}
\newcommand{\collocPointVecArg}[2]{{\boldsymbol \zeta}^{#1,#2}}

\newcommand{\nsamples}{n_s}

\newcommand{\timeWindow}{\tau}
\newcommand{\timeWindowArg}[2]{\tau^{#1,#2}}

\newcommand{\bz}{\boldsymbol 0}
\newcommand{\nsteps}{N_{\tau}}
%\newcommand{\nstepsArg}[1]{\nsteps(#1)}
\newcommand{\nstepsArg}[1]{\nsteps^{#1}}
\newcommand{\lspgWeighting}{\stweightingMatOne}
\newcommand{\lspgWeightingArg}[1]{\stweightingMatOne}
\newcommand{\lspgWeightingST}{\stweightingMatOne_{\text{ST}}}
\newcommand{\lspgWeightingSTArg}[1]{\stweightingMatOne_{\text{ST}}}

%%States
\newcommand{\approxstate}{\tilde{\boldsymbol x}} 
\newcommand{\approxstateDiscrete}{\tilde{\mathbf{x}}} 
\newcommand{\approxstateDiscreteArg}[1]{\approxstateDiscrete^{#1}} 

\newcommand{\approxstateLSPG}{\approxstateDiscrete_{\text{L}}} 

\newcommand{\approxstateArg}[2]{\approxstate^{#1}(#2)} 
\newcommand{\approxstateArgnt}[1]{\approxstate^{#1}} 

\newcommand{\approxstatecolloc}{\overline{\state}}
\newcommand{\approxstatecollocArg}[3]{\overline{\state}^{#1,#2}(#3)}
\newcommand{\statecolloc}{\overline{\state}}
\newcommand{\statecollocArg}[3]{\statecolloc^{#1,#2}(#3)}
\newcommand{\genstateycolloc}{\overline{\statey}}
\newcommand{\genstateycollocArg}[3]{\genstateycolloc^{#1,#2}(#3)}
\newcommand{\genstateycollocArgnt}[2]{\genstateycolloc^{#1,#2}}
\newcommand{\genstatecolloc}{\overline{\overline{\state}}}
\newcommand{\genstatecollocArg}[3]{\genstatecolloc^{#1,#2}(#3)}
\newcommand{\genstatecollocArgnt}[2]{\genstatecolloc^{#1,#2}}
\newcommand{\stateMat}{\mathbf {x}}
\newcommand{\stateMaty}{\mathbf{y}}
\newcommand{\genstatecollocMat}{\overline{\stateMat}}
\newcommand{\genstatecollocMaty}{\overline{\stateMaty}}
\newcommand{\genstatecollocMatyArg}[1]{\overline{\stateMaty}^{#1}}
\newcommand{\genCollocMat}{\hat{\stateMat}}
\newcommand{\genCollocMatArg}[2]{\genCollocMat^{#1,#2}}
\newcommand{\genCollocMaty}{\hat{\stateMaty}}
\newcommand{\genCollocMatyArg}[2]{\genCollocMaty^{#1,#2}}

\newcommand{\collocMatyArg}[2]{{\stateMaty}^{#1,#2}}

\newcommand{\genstatecollocMatArg}[2]{\genstatecollocMat^{#1,#2}}


%% SpaceTime
\newcommand{\spaceTrialSpace}{\mathcal{V}}
\newcommand{\timeTrialSpace}{\mathcal{Y}}
\newcommand{\timeTrialSpaceDiscrete}{\mathcal{Y}_{\text{D}}}

\newcommand{\timeTrialSpaceArg}[1]{\timeTrialSpace^{#1}}


\newcommand{\genstatecollocMatSlab}{\hat{\overline{\overline{\stateMat}}}}
\newcommand{\genstatecollocMatSlabArg}[1]{\genstatecollocMatSlab^{#1}}
\newcommand{\genstatecollocMatySlab}{\hat{\overline{\overline{\stateMaty}}}}
\newcommand{\genstatecollocMatySlabArg}[1]{\genstatecollocMatySlab^{#1}}

\newcommand{\timeSpace}{\mathcal{T}}
\newcommand{\onesFunction}{\mathcal{O}}
\newcommand{\onesFunctionArg}[1]{\mathcal{O}^{#1}}

\newcommand{\timeSpaceArg}[1]{\timeSpace^{#1}}

\newcommand{\stateArg}[2]{\state^{#1}(#2)}

\newcommand{\stateFOM}{{\state}}
\newcommand{\stateFOMIC}{{\stateFOMDiscrete_0}}
\newcommand{\approxstateDiscreteIC}{{\approxstateDiscrete_0}}

\newcommand{\stateFOMArg}[2]{\stateFOM^{#1}(#2)}
\newcommand{\stateFOMArgnt}[1]{\stateFOM^{#1}}

\newcommand{\stateFOMDiscrete}{{\mathbf{x}}}
\newcommand{\stateFOMDiscreteArg}[1]{\stateFOMDiscrete^{#1}}


\newcommand{\stateyArg}[2]{\statey^{#1}(#2)}
\newcommand{\stateArgnt}[1]{\state^{#1}}
\newcommand{\stgenstate}{\hat{\mathbf{\overline{x}}}}
\newcommand{\stgenstateArg}[1]{\stgenstate^{#1}}
\newcommand{\stgenstateStar}{\hat{\mathbf{\overline{x}}}_*}
\newcommand{\stgenstateStarArg}[1]{\stgenstateStar^{#1}}

\newcommand{\stgenstatey}{\hat{\mathbf{\overline{y}}}}
\newcommand{\stgenstateyArg}[1]{\stgenstatey^{#1}}

\newcommand{\stateyDiscrete}{\mathbf{y}}
%\newcommand{\stateyDotDiscrete}{\mathbf{\dot{y}}}
\newcommand{\stateyDotDiscrete}{\mathbf{{v}}}


\newcommand{\statey}{\boldsymbol y}
\newcommand{\stateIntercept}{\mathbf{x}_\text{ref}}
\newcommand{\stateInterceptArg}[1]{\stateIntercept^{#1}}
\newcommand{\stateInterceptST}{\mathbf{x}_\text{st}}
\newcommand{\stateInterceptSTArg}[1]{\stateInterceptST^{#1}}

\newcommand{\stateInterceptMat}{{\stateMat_*}^{n,i}}

\newcommand{\genstatey}{\hat{\boldsymbol y}}
\newcommand{\genstateyDiscrete}{\hat{\mathbf{y}}}

\newcommand{\decoderColloc}{\decoder_c^{n,i}}
\newcommand{\decoderCollocArg}[2]{\decoder_c^{#1,#2}}

\newcommand{\residST}{\overline{\boldsymbol r}}
\newcommand{\genresidST}{\hat{\overline{\boldsymbol r}}}
\newcommand{\solveAlgo}{\underset{\text{solve}}{\leftarrow}}
\newcommand{\residSlabArg}[1]{\overline{\boldsymbol r}^{#1}}
\newcommand{\residLMSSlabArg}[1]{\overline{\overline{\boldsymbol r}}^{#1}_{\text{}}}
%\newcommand{\spatialAcronym}{ST-SDR}
%\newcommand{\spaceTimeAcronym}{ST-STDR}
\newcommand{\spatialAcronym}{S-reduction}
\newcommand{\spaceTimeAcronym}{ST-reduction}
%\newcommand{\spatialAcronym}{SDR}
%\newcommand{\spaceTimeAcronym}{STDR}
\newcommand{\residArg}[1]{\resid^{#1}}
\newcommand{\decoderDotColloc}{\dot{\decoder}_c}
\newcommand{\residCollocArg}[1]{\resid_c^{#1}}
\newcommand{\residLMSArg}[1]{\resid_{}^{#1}}
\newcommand{\lmsSlabArg}[1]{g^{#1}}
\newcommand{\lmsCollocArg}[1]{k^{#1}}
\newcommand{\lmsWidthArg}[2]{k^{#1}(#2)}
\newcommand{\ntimeSteps}{N_t}
\newcommand{\nslabs}{N_s}
\newcommand{\RR}[1]{\mathbb{R}^{#1}}
\newcommand{\RRStar}[2]{\mathbb{V}_{#1}(\RR{#2})}
%velocity args
\newcommand{\veloargsromn}{\basisspaceArg{n} \genstateArg{n}{t} + \stateInterceptArg{n},t}
%\newcommand{\veloargsromnArg}[1]{\basisspaceArg{#1} \genstateArg{#1}{t} + \stateInterceptArg{#1},t}

\newcommand{\veloargsromnt}{\basisspace \genstateArgnt{n} + \stateInterceptArg{n},t}
\newcommand{\veloargsromntArg}[1]{\basisspace \genstateArgnt{n}_{#1} + \stateInterceptArg{n},t}

\newcommand{\veloargs}{\stateArg{}{t}}

%\newcommand{\timeStartArg}[1]{t_s(#1)}
\newcommand{\timeStartArg}[1]{t_s^{#1}}
%\newcommand{\timeEndArg}[1]{t_f(#1)}
\newcommand{\timeEndArg}[1]{t_f^{#1}}
\newcommand{\trialspace}{\mathcal{V } } 
%\newcommand{\trialspaceArg}[1]{\trialspace^{#1}} 
\newcommand{\trialspaceArg}[1]{\trialspace^{#1}} 

\newcommand{\stspace}{\mathcal{ ST } } 
\newcommand{\stspaceS}{\mathcal{ ST }_{\text{S}} } 
\newcommand{\stspaceST}{\mathcal{ ST }_{\text{ST}} } 

\newcommand{\stspaceDiscrete}{\mathcal{ ST }_{\text{D}} } 


\newcommand{\stspaceBoundStartArg}[1]{\stspaceBoundStart^{#1}} 
\newcommand{\stspaceBoundStart}{\mathcal{B}_{s}} 
\newcommand{\stspaceBoundEndArg}[1]{\stspaceBoundEnd^{#1}} 
\newcommand{\stspaceBoundEnd}{\mathcal{B}_{e}} 

\newcommand{\stspaceBoundArg}[1]{\stspaceBound^{#1}} 
\newcommand{\stspaceBound}{\mathcal{B}\timeSpace} 
\newcommand{\stspaceArg}[1]{\stspace^{#1}} 
\newcommand{\stspaceSTArg}[1]{\stspaceST^{#1}} 
\newcommand{\stspaceSArg}[1]{\stspaceS^{#1}} 
\newcommand{\stbasis}{\boldsymbol \Pi}
\newcommand{\stbasisVec}{\boldsymbol \pi}
\newcommand{\stbasisVecArg}[2]{\boldsymbol \pi^{#1}_{#2}}

\newcommand{\stbasisArg}[2]{\stbasis^{#1}(#2)}
\newcommand{\stbasisArgnt}[1]{\stbasis^{#1}}
\newcommand{\stbasisDot}{\dot{\boldsymbol \Pi}}
\newcommand{\stbasisDotArg}[2]{\stbasisDot^{#1}(#2)}
\newcommand{\stbasisDotArgnt}[1]{\stbasisDot^{#1}}

\newcommand{\genstateIC}{ \basisspaceTArg{}( \stateFOMIC - \stateInterceptArg{}) }
\newcommand{\genstateICOne}{ [\basisspaceArg{1}]^T( \stateFOMIC -
\stateInterceptArg{1}) }

\newcommand{\elltwo}{\ell^2}
%dimensions
\newcommand{\romdim}{K}
\newcommand{\romdimArg}[1]{K^{#1}}
\newcommand{\fomdim}{N}

%basis
\newcommand{\basisspace}{\basismat}
\newcommand{\basisspaceArg}[1]{\basismat^{#1}}
\newcommand{\basisspaceTArg}[1]{[\basismat^{#1}]^T}

\newcommand{\basismat}{\mathbf {V}}
\newcommand{\basisvec}{\mathbf{v}}
\newcommand{\basismatArg}[1]{\basismat^{#1}}
\newcommand{\basisvecArg}[1]{\basisvec^{#1}}

\newcommand{\minintegrand}{\mathcal{I}}
\newcommand{\minintegrandColloc}{\mathcal{I}_c}

\newcommand{\minintegrandMat}{\boldsymbol I}
\newcommand{\minintegrandMatArg}[1]{\minintegrandMat^{#1}}

%% Euler Lagrange
\newcommand{\integrand}{\mathcal{I}}

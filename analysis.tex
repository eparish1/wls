\section{Analysis}\label{sec:analysis}
\KTC{HERE}
This section provides theoretical analysis of the \methodAcronym\ \approachKwd. First, we demonstrate equivalence conditions 
between \methodAcronym\  with (uniform) \spatialAcronym\ trial subspaces and
the Galerkin ROM in the limit $\DeltaSlabArg{n} \rightarrow 0$.
%\item For direct methods leveraging uniform quadrature for discretization of the objective functional, the \methodAcronym\ approach recovers the LSPG approach when 
%the time slab corresponds to a single time-step. 
%\end{enumerate}
Next, we derive \textit{a priori} error bounds for autonomous systems. 
\subsection{Equivalence conditions}
\begin{theorem}\label{theorem:galerkin_equiv}\textit{(Galerkin equivalence)}
For sequential minimization over infinitesimal time windows and uniform
	\spatialAcronym\ trial spaces, i.e., $\trialspaceArg{n} = \trialspace$,
	$n=1,\ldots,\nslabs$, the \methodAcronym\ approach (weakly) recovers
	Galerkin projection.
\end{theorem}
\begin{proof}
%\begin{equation}
% \genstate^n(t) = \underset{\genstatey}{\text{argmin }} \mathcal{J}^n(\decoder(\genstatey) ) , \qquad n = 1,2,\ldots,\nslabs,
%\end{equation}
The \methodAcronym\ approach with uniform \spatialAcronym\ subspaces comprises solving the following sequence of minimization problems for $\genstateArgnt{n}$, $n=1,\ldots,\nslabs$,
\begin{equation}\label{eq:obj_proof}
\begin{split}
      & \underset{\genstateyArgnt{} \in \RR{\romdim} \otimes \timeSpaceArg{n}}{\text{minimize}}\; \mathcal{J}^n(\basisspace \genstateyArgnt{} + \stateIntercept \otimes \onesFunctionArg{n}), \\ 
      & \text{subject to }\; \genstateyArg{}{\timeStartArg{n}} =
\begin{cases} \genstateArg{n-1}{\timeEndArg{n-1}} & n = 2,\ldots,\nslabs \\
\genstateIC & n=1. \end{cases} 
\end{split}
\end{equation}
Following the derivation of the Euler--Lagrange equations presented in Appendix~\ref{appendix:eulerlagrange} leads to Eq.~\eqref{eq:euler_lagrange_analysis}, which in this context comprises the sequence of systems to be solved for $\genstate^n$ (and, implicitly, $\genstateDotArgnt{n}$) over $t \in [\timeStartArg{n},\timeEndArg{n}]$:
\begin{multline}\label{eq:g_equiv_1}
 \int_{\timeStartArg{n}}^{\timeEndArg{n}} \bigg( \frac{\partial \minintegrandArg{n}  }{\partial \genstateyDiscrete }(\genstateArg{n}{t},\genstateDotArg{n}{t},t)  \variationArgntt{n}{t}  - \frac{d}{dt}\bigg( \frac{\partial \minintegrandArg{n}}{\partial \genstateyDiscreteDot} (\genstateArg{n}{t},\genstateDotArg{n}{t},t ) \bigg) \variationArgntt{n}{t} \bigg)dt +\\ \bigg(\frac{\partial \minintegrandArg{n}}{\partial \genstateyDiscreteDot}(\genstateArg{n}{\timeEndArg{n}},\genstateDotArg{n}{\timeEndArg{n}},\timeEndArg{n}) \bigg) \variationArgntt{n}{\timeEndArg{n}}   = 0,
\end{multline}
with the boundary conditions
\begin{equation*}
 \genstate^n(\timeStartArg{n})  = 
\begin{cases}
\genstate^{n-1}(\timeEndArg{n-1}) & n=2,\ldots,\nslabs, \\
\genstateIC & n=1. \end{cases}  
%\bigg[\frac{\partial \minintegrandArg{n}}{\partial \genstateyDiscreteDot}(\genstateArg{n}{\timeEndArg{n}},\genstateDotArg{n}{\timeEndArg{n}},\timeEndArg{n}) \bigg]^T= \boldsymbol 0.
\end{equation*}
In the above, $\variationArgn{n}: [\timeStartArg{n},\timeEndArg{n}] \rightarrow \RR{\romdim}$ is an an arbitrary function that satisifes $\variationArgntt{n}{\timeStartArg{n}} = \bz$. 
%$$\genstate^n(\timeStartArg{n}) = \genstate^{n-1}(\timeEndArg{n-1} ), \qquad \frac{\partial \integrand}{\partial \dot{\genstate}^n} \bigg|_{t=\timeEndArg{n}} = \boldsymbol 0 $$
To examine what happens when the window size shrinks, take a uniform window size and let $\timeEndArg{n} = \timeStartArg{n} + \zeta$. Note that $\timeStartArg{n} = \zeta (n-1)$ and $\timeEndArg{n} = \zeta n$. %We have, 
%\begin{equation}\label{eq:g_equiv_1}
% \int_{\zeta (n-1)}^{\zeta n} \bigg(\bigg[ \frac{\partial \minintegrandArg{n}  }{\partial \genstateyDiscrete} (\genstateArg{n}{t},\genstateDotArg{n}{t},t)  \bigg] \variationArgntt{n}{t}  - \bigg[ \frac{\partial \minintegrandArg{n}}{\partial \genstateyDiscreteDot} (\genstateArg{n}{t},\genstateDotArg{n}{t},t) \bigg] \variationArgntt{n}{t} \bigg)dt= 0, \qquad n = 1,\ldots,\nslabs.
%\end{equation}
Noting that $\variationArgn{n}$ is an arbitrary function, we take the limit $\zeta \rightarrow 0^+$ to collapse the window size and obtain the (infinite) sequence of problems
\begin{equation*}
 \genstateArg{n}{\zeta(n-1)} = 
\begin{cases}
\genstate^{n-1}(\zeta(n-1)) & n=2,\ldots,\nslabs,\\
\genstateIC & n=1, \end{cases} \qquad
%\genstate^{n-1}(\zeta n), \qquad \frac{\partial \integrand}{\partial \dot{\genstate}^n} \bigg|_{t= \zeta n} = \boldsymbol 0, \qquad n = 1,2,\nslabs,
\bigg[ \frac{\partial \minintegrandArg{n}}{\partial \genstateyDiscreteDot}(\genstateArg{n}{\zeta n},\genstateDotArg{n}{\zeta n},\zeta n) \bigg]^T= \boldsymbol 0,
\end{equation*}
where it is noted that the integral in Eq.~\eqref{eq:g_equiv_1} is automatically satisfied as
 \begin{equation*}
\lim_{\zeta \rightarrow 0^+}  \int_{\zeta (n-1)}^{\zeta n} h(t) dt = 0
\end{equation*}
for any continuous $h(t)$.
%subject to $\genstate^n(\timeStartArg{n}) = \genstate^{n-1}(\timeEndArg{n-1} )$. 
Noting that the derivative evaluates to
\begin{equation*}
\bigg[ \frac{\partial \minintegrandArg{n}}{\partial \genstateyDiscreteDot}(\genstateArg{n}{\zeta n},\genstateDotArg{n}{\zeta n},\zeta n) \bigg]^T=
\basisspace^T \stweightingMatArgt{}{\zeta n} \basisspace \genstateDotArg{n}{\zeta n} -  \basisspace^T \stweightingMatArgt{}{\zeta n} \velocity(\basisspace \genstateArg{n}{\zeta n} + \stateIntercept ,\zeta n), 
%\bigg[\frac{\partial \minintegrand}{\partial \dot{ \genstate } } \bigg]^T =  \basisspace^T \stweightingMat \basisspace \frac{d}{dt}{\genstate} -  \basisspace^T \stweightingMat \velocity(\decoder(\genstate)), 
\end{equation*}
we then have 
\begin{equation*}
\basisspace^T \stweightingMatArgt{}{\zeta n} \basisspace \genstateDotArg{n}{\zeta n} -  \basisspace^T \stweightingMatArgt{}{\zeta n} \velocity(\basisspace \genstateArg{n}{\zeta n} + \stateIntercept ,\zeta n) = \bz, \qquad n=1,\ldots,\nslabs, 
\end{equation*}
with the boundary conditions  $\genstateArg{n}{\zeta (n-1)} = \genstateArg{n-1}{\zeta(n-1)}$ for $n=2,\ldots,\nslabs$ and $\genstateArg{1}{0} = \genstateIC $. %Left multiplying by $\mass^{-1}$ yields,
%\begin{equation*}
%\genstateDotArg{n}{\timeEndArg{n}} -  \mass^{-1} \basisspace^T \stweightingMatArg{n} \velocity(\basisspace \genstateArg{n}{\timeEndArg{n}} + \stateIntercept ,\timeEndArg{n}) = \bz, \qquad n=1,2,\ldots,\nslabs. 
%\end{equation*}
%\begin{equation*}
%\bigg[  \frac{d}{dt}{\genstate^n} -  \basisspace^T  \velocity(\decoder(\genstate^n)) \bigg]_{t=n\zeta}= \boldsymbol 0, \qquad n = 1,2,\nslabs,
%\end{equation*}
%subject to $\genstate^n( n\zeta) = \genstate^{n-1}(n\zeta)$. 
In the limit of $\zeta \rightarrow 0$ (and hence $\nslabs \rightarrow \infty$) this is a (weak) statement of the Galerkin ROM.\footnote{Formally, the continuous time domain $[0,T]$ with $T \in \RR{+}$ cannot be recovered as the set $n=\{1,\ldots,\infty\}$ is an \textit{infinitely countable set}, while $\RR{+}$ is uncountable.}
%where at every time-instance the system obeys,
%\begin{equation*}
% \genstateDotArg{n}{t} - \mass^{-1} \basisspace^T \stweightingMatArg{n} \velocity(\basisspace \genstateArg{n}{t} + \stateIntercept ,t) = \bz, \qquad n=1,2,\ldots,\nslabs. 
%\end{equation*}
\end{proof}
%
%\begin{theorem}\label{theorem:LSPG_RK_equiv}\textit{(LSPG equivalence for Collocation Schemes)}
%For direct methods leveraging a collocation scheme and uniform quadrature for discretization of the objective~\eqref{eq:obj_colloc}, the \methodAcronym\ approach recovers the LSPG approach when the time slab corresponds to a single time-step and no weighting is used; i.e., $\stweightingMatArgt{n}{t} = \mathbf{I}$.\footnote{Ref.~\cite{carlberg_lspg_v_galerkin} derives the LSPG method for the case where the weighting matrix is a function of the state. As this is not considered here, we forgo equivalence conditions for the 
%case with hyper-reduction.}
%\end{theorem}
%\begin{proof}
%Decomposing each time slab into one time-step  instance and following Ref.~\cite{carlberg_lspg_v_galerkin} Sect. 4.1.2, for collocation (Runge-Kutta) schemes the LSPG-ROM solves the following minimization problem over first and only time-step instance on the $n$th time slab,
%\begin{align*}
%\collocMatArg{n}{1}
% = \underset{\collocMatyArg{n}{1} \in \trialspace}{\text{argmin } }
%\begin{bmatrix} \residCollocArg{n,1} ( \collocMatyArg{n}{1} ) \end{bmatrix}^T
%\begin{bmatrix} \residCollocArg{n,1} ( \collocMatyArg{n}{1} )  \end{bmatrix}.
%\end{align*}
%%where $\mathbf{1} = \begin{bmatrix} 1 & \ldots  & 1 \end{bmatrix}^T \in \RR{\fomdim \ncollocArg{n}{1}}$. 
%The LSPG-ROM is seen to uniformly minimize the residual at 
%every stage (e.g., collocation point) of the scheme. 
%
%Setting the number of time-step instances over the $n$th slab to be one in Eqs.~\eqref{eq:obj_colloc}  and~\eqref{eq:min_colloc}, 
%the \methodAcronym-ROM solves the minimization problem,
%\begin{equation*}
%\collocMatArg{n}{1} = \underset{\collocMatyArg{n}{1}  \in \trialspace}{\text{argmin } }
%\objectiveArgC{n} (\collocMatyArg{n}{1}),
%\end{equation*}
%where,
%\begin{align*}
%\objectiveArgC{n} &: (\collocMatyArg{n}{1}) \mapsto 
%\frac{1}{2}  [ \residCollocArg{n,1} ( \collocMatyArg{n}{1} ) \circ \sqrt{\quadWeightsVecArg{n,1}}]^T  [ \residCollocArg{n,1}  ( \collocMatyArg{n}{1} ) \circ \sqrt{\quadWeightsVecArg{n,1 }}] \\
%&: \RR{\fomdim} \otimes \RR{\collocOrderArg{n}{i} \nstepsArg{n}} \rightarrow \RR{} ,
%\end{align*}
%where $\quadWeightsVecArg{n,1} \in \RR{\fomdim \ncollocArg{n}{1}}$ are quadrature weights. Setting $\quadWeightsVecArg{n,1} = \mathbf{1}$, the \methodAcronym-ROM is seen 
%to solve the same minimization problem as the LSPG-ROM. 
%\end{proof}
%
%\begin{remark}
%Theorem~\ref{theorem:LSPG_RK_equiv} demonstrated equivalence conditions between the \methodAcronym-ROM and the LSPG-ROM for direct collocation methods. 
%With respect to linear multistep methods, due to the assumption made in Remark~\ref{remark:LMS}, in where it is assumed for  simplicity that the 
%linear multistep method employs time-instances only within the current time slab, equivalence conditions for linear multistep methods are not presented here; 
%such conditions require employing linear  multistep methods that include time-instances from (potentially multiple) previous time slabs.
%It is, however, straightforward to observe that in this case equivalence conditions similar to those presented in Theorem~\ref{theorem:LSPG_RK_equiv} exist.
%\end{remark} 

\subsection{\textit{A priori} error bounds}
\textit{A priori} error bounds are derived for \spatialAcronym\ trial subspaces in the case that the velocity is autonomous, $\velocity(\cdot,t) \equiv \velocity(\cdot)$, and no weighting matrix is employed, $\stweightingMat = \mathbf{I}$.
%We define the residual as,
%$$\resid(\state) = \dot{\state} - \velocity(\state).$$
In the following,  $\stateFOMSolArg{n}$ is the state implicitly defined as the solution to full-order model, $\stateROMSolArg{n}$ is the state implicitly defined as the solution to the \methodAcronym-ROM, $\adjointROMSolArg{n}$ is the costate solution to the \methodAcronymROM, and the error is defined as
$$\errorArg{n} \defeq \stateFOMSolArg{n} - \stateROMSolArg{n}.$$
Additionally, we denote $\stateFOMProjSolArg{n}$, $n=1,\ldots,\nslabs$ to be the $\elltwo$ optimal solution
$$\stateFOMProjSolArg{n} = \underset{\statey \in
\stspace^n}{\text{arg}\,\text{min } } \intSlabArg{n} \norm{ \statey(t) - \stateFOMSolArgt{n}{t} }^2.$$ 
We begin by stating assumptions that will be used in the subsequent derivations.

\begin{itemize}
\item \textbf{A1:} We assume  Lipshitz continuity on the residual:
$$ \norm{\resid(\statew,\timeDummy) - \resid(\statey,\timeDummy) } \le \lipshitz \norm{\statew(\timeDummy) - \statey(\timeDummy)},$$
where  
\begin{align*}
\resid &: \, (\statey,\timeDummy) \mapsto \dot{\statey}(\timeDummy) - \velocity(\stateyArg{}{\timeDummy},\timeDummy) ,\\
&: \, \RR{N} \otimes \timeSpaceArg{} \times [0,T] \mapsto \RR{N}.
\end{align*}

\item \textbf{A2:} We assume inverse Lipshitz continuity on the integrated residual for two trajectories starting from the FOM solution over the $n$th window:
$\forall \statew,\statey \in \stspaceArg{n}_*, \stspaceArg{n}_* = \{ \statew \in \RR{\fomdim} \otimes \timeSpaceArg{n} | \statew(\timeStartArg{n}) = \stateFOMArg{}{\timeStartArg{n}} \}$ we assume 
%$\forall \statew,\statey$ with $\statew(\timeStartArg{n} ) = \statey(\timeStartArg{n})$ \EP{Put this in  a set maybe? I think that this, along with the trajectory being ''sufficiently smooth", should be enough to avoid the case where we have two different trajectories that exactly satisfy the residual. Something like $\statew,\statey \in \stspaceArg{n}_*, \stspaceArg{n}_* = \{ \statew \in \stspaceArg{n} | \statew(\timeStartArg{n}) = \mathbf{g} \}$ for arbitrary $\mathbf{g}$},
$$  \intSlabArg{n} \norm{\statew(t) - \statey(t)} dt \le  \lipshitziArg{n} \intSlabArg{n} \norm{\resid(\statew,t) - \resid(\statey,t) } dt.$$
\item \textbf{A3:} We assume that the FOM solution at the start of each window lies within the range of the trial subspace:
$$ \stateFOMSolArgt{n}{\timeStartArg{n}} \in \trialspaceArg{n} + \stateInterceptArg{n}.$$
\end{itemize} 
%A3: We assume Lipshitz continuity on the velocity
%$$ \norm{\velocity(\state) - \velocity(\boldsymbol y) } \le \lipshitzv \norm{\state - \boldsymbol y}.$$
%
%A4: We assume,
%$$ \norm{ \stateROMSolArgt{n}{\timeStartArg{n}} - \stateFOMSolArgt{n}{\timeStartArg{n}} } \le \constAInt \intSlabArg{n-1} \norm{ \errorArg{n-1} }dt.$$

\begin{theorem}(\textit{A priori} error bounds)\label{theorem:apriori_bound}
Error bounds for \methodAcronymROMs\ over the $n$th window with \spatialAcronym\ trial subspaces are:
\begin{equation}\label{eq:apriori_bound}
\intSlabArg{n} \norm{\errorArgt{n}{t}} dt \le \DeltaSlabArg{n} \sqrt{1 + \lipshitz} \norm{ \errorArgt{n}{\timeStartArg{n}}  }   + \lipshitziArg{n} \intSlabArg{n} \norm{ \resid(\stateFOMProjSolArg{n},t) } dt.
\end{equation}
\EP{We can do over the entire time domain, but we have to introduce an assumption to relate $ \errorArgt{n}{\timeStartArg{n}} $ to $\intSlabArg{n-1} \norm{\errorArgt{n-1}{t}} dt$}
%\begin{equation}\label{eq:apriori_bound}
%\sum_{i=1}^{\nslabs} \intSlabArg{i} \norm{\errorArgt{i}{t}} dt \le \sum_{i=1}^{\nslabs} \DeltaSlabArg{i} (1 + \lipshitz) \norm{\errorArgt{i}{\timeStartArg{i}} }   + \lipshitzi \int_0^T \norm{ \resid(\stateFOMProjSolArg{n},t) } dt.
%\end{equation}
\end{theorem}
\begin{proof}
To obtain an error bound over the $n$th window, we must account for the 
fact that the initial conditions into the $n$th window can be incorrect. To this end, 
we start by defining a new quantity, $\stateROMStarSolArg{n}$, $n=1,\ldots,\nslabs$ to implicitly be the solution to the minimization problem 
%with the correct initial conditions
\begin{equation}\label{eq:min_correct}
\begin{split}
& \underset{\statey \in \stspaceArg{n}}{\text{minimize } }
\objectiveArg{n}(\statey),\\
& \text{subject to } \statey(\timeStartArg{n})= \stateFOMSolArgt{n}{\timeStartArg{n}}.
\end{split}
\end{equation}
Note that minimization problem~\eqref{eq:min_correct} is is equivalent to the \methodAcronym\ minimization problem~\eqref{eq:tclsrm}, but uses the FOM solution for the initial conditions. 
Additionally, define $\adjointROMStarSolArg{n}$, $n=1,\ldots,\nslabs$ to be the associated costate solution.
 The error in the solution obtained by the 
\methodAcronymROM\ over the $n$th window at time $t$ can be written as
\begin{equation*}
\norm{ \stateROMSolArgt{n}{t} - \stateFOMSolArgt{n}{t}} = 
\norm{\stateROMSolArgt{n}{t} - \stateROMStarSolArgt{n}{t} + \stateROMStarSolArgt{n}{t} -  \stateFOMSolArgt{n}{t} }.
\end{equation*}
Applying triangle inequality yields
\begin{equation*}
\norm{ \stateROMSolArgt{n}{t} - \stateFOMSolArgt{n}{t}} \le 
\norm{\stateROMSolArgt{n}{t} - \stateROMStarSolArgt{n}{t}} + \norm{ \stateROMStarSolArgt{n}{t} -  \stateFOMSolArgt{n}{t} }.
\end{equation*}
%$$\norm{ \state_R - \state_f} = \norm{ \state_R - \state^*_R + \state^*_R - \state_f}.$$
%$$ \le \norm{\state_R - \state^*_R} + \norm{\state^*_R - \state_f}.$$
Integrating over the $n$th window and using the definition $\errorArg{n} \defeq \stateFOMSolArg{n} - \stateROMSolArg{n}$ yields
$$\intSlabArg{n} \norm{\errorArgt{n}{t}} dt \le \intSlabArg{n} \norm{\stateROMSolArgt{n}{t} - \stateROMStarSolArgt{n}{t}} dt +  \intSlabArg{n} \norm{\stateROMStarSolArgt{n}{t} - \stateFOMSolArgt{n}{t}}dt.$$
Applying A2 (inverse Lipshitz continuity on the integrated residual), the last term on the right-hand side becomes
\begin{equation*}
\intSlabArg{n} \norm{\errorArgt{n}{t}} dt \le \intSlabArg{n} \norm{\stateROMSolArgt{n}{t} - \stateROMStarSolArgt{n}{t}} dt +  \lipshitziArg{n} \intSlabArg{n} \norm{ \resid(\stateROMStarSolArg{n},t) - \resid(\stateFOMSolArg{n},t)} dt.
\end{equation*}
Noting that, $\forall t \in [0,T]$, we have $\resid(\stateFOMSolArg{n},t) = \bz$ and get the simplification 
%As $\resid{\state_f} = \boldsymbol 0$,
\begin{equation*}
\intSlabArg{n} \norm{\errorArgt{n}{t}} dt \le \intSlabArg{n} \norm{\stateROMSolArgt{n}{t} - \stateROMStarSolArgt{n}{t}} dt + \lipshitziArg{n} \intSlabArg{n} \norm{ \resid(\stateROMStarSolArg{n},t) } dt.
\end{equation*}
%$$\int \norm{\error} dt \le \int \norm{\state_R - \state^*_R} dt +  \int \norm{\resid{\state^*_R}}dt.$$
Leveraging the residual-minimization property of \methodAcronym, we have 
$$ \intSlabArg{n} \norm{\resid(\stateROMStarSolArg{n},t)}dt \le \intSlabArg{n} \norm{\resid(\stateFOMProjSolArg{n},t)}dt.$$
%$$ \int \norm{\resid{\state^*_R}}dt \le \int \norm{\resid{\tilde{\state}_f}}dt,$$
This leads to the following expression for the error over the $n$th window,
\begin{equation}\label{eq:boundtmp}
\intSlabArg{n} \norm{\errorArgt{n}{t}} dt \le \intSlabArg{n} \norm{\stateROMSolArgt{n}{t} - \stateROMStarSolArgt{n}{t}} dt + \lipshitziArg{n} \intSlabArg{n} \norm{ \resid(\stateFOMProjSolArg{n},t) } dt.
\end{equation}
%$$\int \norm{\error} dt \le \int \norm{\state_R - \state^*_R} dt +   \int \norm{\resid{\tilde{\state}_f}}dt.$$
\begin{comment}
To obtain an expression for $\norm{\stateROMSolArg{n}{t} - \stateROMStarSolArgt{n}{t}}$ start by noting that
$\stateROMStarSolArg{n}$ and $\stateROMSolArg{n}$ are defined by the systems,
\begin{align*} 
&\frac{d}{dt} \genstateROMSolArg{n}   -
\basisspace^T  \velocity(\basisspace \genstateROMSolArg{n} + \stateIntercept,t) =  \adjointROMSolArg{n} , \\
%\end{equation} \begin{equation}\label{eq:lspg_adjoint}
 &\frac{d}{dt} \adjointROMSolArg{n} + \basisspace^T \bigg[\frac{\partial
\velocity}{\partial \stateyDiscrete}(\basisspace \genstateROMSolArg{n} +
\stateIntercept,t)\bigg]^T \basisspace \adjointROMSolArg{n} = \basisspace^T \bigg[
\frac{\partial \velocity}{\partial \stateyDiscrete} (\basisspace \genstateROMSolArg{n} +
\stateIntercept,t) \bigg]^T \bigg( \mathbf{I} -   \basisspace \basisspace^T
\bigg)    \velocity(\basisspace \genstateROMSolArg{n} + \stateIntercept,t) , \\  
& \genstateROMSolArgt{n}{\timeStartArg{n}} =
\begin{cases} \genstateROMSolArgt{n-1}{\timeEndArg{n-1}} & n=2,\ldots,\nslabs, \\
\basisspace^T(\approxstateIC - \stateIntercept) & n=1, \end{cases}\\
&\adjointROMSolArgt{n}{\timeEndArg{n}} = \boldsymbol 0 .  \end{align*}
\begin{align*} 
&\frac{d}{dt} \genstateROMStarSolArg{n}   -
\basisspace^T  \velocity(\basisspace \genstateROMStarSolArg{n} + \stateIntercept,t) =  \adjointROMStarSolArg{n} , \\
%\end{equation} \begin{equation}\label{eq:lspg_adjoint}
 &\frac{d}{dt} \adjointROMStarSolArg{n} + \basisspace^T \bigg[\frac{\partial
\velocity}{\partial \stateyDiscrete}(\basisspace \genstateROMStarSolArg{n} +
\stateIntercept,t)\bigg]^T \basisspace \adjointROMStarSolArg{n} = \basisspace^T \bigg[
\frac{\partial \velocity}{\partial \stateyDiscrete} (\basisspace \genstateROMStarSolArg{n} +
\stateIntercept,t) \bigg]^T \bigg( \mathbf{I} -   \basisspace \basisspace^T
\bigg)    \velocity(\basisspace \genstateROMStarSolArg{n} + \stateIntercept,t) , \\  
& \genstateROMStarSolArgt{n}{\timeStartArg{n}} =
\begin{cases} \genstateROMStarSolArgt{n-1}{\timeEndArg{n-1}} & n=2,\ldots,\nslabs, \\
\basisspace^T(\approxstateIC - \stateIntercept) & n=1, \end{cases}\\
&\adjointROMStarSolArgt{n}{\timeEndArg{n}} = \boldsymbol 0 .  \end{align*}
%Next, as the codomain of the squared $2$-norm is $\RR{+}$, one has, 
%$$\norm{\genstateROMSolArg{n} - \genstateROMStarSolArg{n}}^2 \le  \norm{\genstateROMSolArg{n} - \genstateROMStarSolArg{n}}^2 + \norm{\adjointROMSolArg{n} - 
%\adjointROMStarSolArg{n} }^2 .$$ 
\end{comment}
We now find an upper bound for $\intSlabArg{n} \norm{\stateROMSolArgt{n}{t} - \stateROMStarSolArgt{n}{t}} dt$. 
First we can bound $\norm{\stateROMSolArgt{n}{t} - \stateROMStarSolArgt{n}{t}}^2$ by above:
\begin{align*}
\norm{\stateROMSolArgt{n}{t} - \stateROMStarSolArgt{n}{t}}^2 &\le  \norm{\stateROMSolArgt{n}{t} - \stateROMStarSolArgt{n}{t}}^2 + \norm{ \adjointROMSolArgt{n}{t} - 
\adjointROMStarSolArgt{n}{t} }^2 , \\
&\le  \norm{\genstateROMSolArgt{n}{t} - \genstateROMStarSolArgt{n}{t}}^2 + \norm{ \adjointROMSolArgt{n}{t} - 
 \adjointROMStarSolArgt{n}{t} }^2, 
\end{align*}
where $\genstateROMSolArg{n}$ and $\genstateROMStarSolArg{n}$ are the generalized coordinates of $\stateROMSolArg{n}$ and $\stateROMStarSolArg{n}$ and it is additionally noted that 
$\norm{\genstateROMSolArgt{n}{t} - \genstateROMStarSolArgt{n}{t}} = 
 \norm{\stateROMSolArgt{n}{t} - \stateROMStarSolArgt{n}{t}}$. Taking the square root yields
\begin{align*}
\norm{\stateROMSolArgt{n}{t} - \stateROMStarSolArgt{n}{t}} &\le \sqrt{ \norm{\genstateROMSolArgt{n}{t} - \genstateROMStarSolArgt{n}{t}}^2 + \norm{ \adjointROMSolArgt{n}{t} - 
 \adjointROMStarSolArgt{n}{t} }^2}. 
\end{align*}
Critically, we now utilize the fact that the system governing the \methodAcronymROM\ with \spatialAcronym\ trial spaces is Hamiltonian. 
Applying Liouville's theorem for Hamiltonian systems, we have the requirement 
%states that, for Hamiltonian systems, ``the volume enclosed by a closed surface in phase-space is 
%constant as that volume moves through phase-space~\cite{liouville}.'' In the present context, Liouville's theorem states, 
$$ \frac{d}{dt}  \sqrt{ \norm{\genstateROMSolArg{n} - \genstateROMStarSolArg{n}}^2 + \norm{\adjointROMSolArg{n} - 
\adjointROMStarSolArg{n} }^2}  = \bz \qquad t \in [\timeStartArg{n},\timeEndArg{n}].$$
Thus, we have
\begin{align*}
\intSlabArg{n} \norm{\stateROMSolArgt{n}{t} - \stateROMStarSolArgt{n}{t}} dt &\le 
\intSlabArg{n} \sqrt{ \norm{\genstateROMSolArgt{n}{t} - \genstateROMStarSolArgt{n}{t}}^2 + \norm{ \adjointROMSolArgt{n}{t} - 
 \adjointROMStarSolArgt{n}{t} }^2 } dt, \\
&\le  \DeltaSlabArg{n}  \sqrt{ \norm{\genstateROMSolArgt{n}{\timeStartArg{n}} - \genstateROMStarSolArgt{n}{\timeStartArg{n} }}^2 + \norm{\adjointROMSolArgt{n}{\timeStartArg{n}} - 
\adjointROMStarSolArgt{n}{\timeStartArg{n}}  }^2}.
\end{align*}
%==================\\
%We now find an upper bound for $\intSlabArg{n} \norm{\stateROMSolArgt{n}{t} - \stateROMStarSolArgt{n}{t}} dt$. 
%First, as the codomain of the $2$-norm is $\RRplus$, we can bound $\norm{\stateROMSolArgt{n}{t} - \stateROMStarSolArgt{n}{t}}$ by above,
%\EP{Something got messed up here in a merge, fixing right now}
%\begin{align*}
%\norm{\stateROMSolArgt{n}{t} - \stateROMStarSolArgt{n}{t}} &\le  \norm{\stateROMSolArgt{n}{t} - \stateROMStarSolArgt{n}{t}} + \norm{\adjointROMSolArgt{n}{t} - 
%\adjointROMStarSolArgt{n}{t} } , \\
%%&\le \norm{\basisspace}  \norm{\genstateROMSolArgt{n}{t} - \genstateROMStarSolArgt{n}{t}} + \norm{\basisspace} \norm{ \adjointROMSolArgt{n}{t} - 
%% \adjointROMStarSolArgt{n}{t} },\\ 
%&\le  \norm{\genstateROMSolArgt{n}{t} - \genstateROMStarSolArgt{n}{t}} + \norm{ \adjointROMSolArgt{n}{t} - 
% \adjointROMStarSolArgt{n}{t} }, 
%\end{align*}
%where $\genstateROMSol$ and $\genstateROMStarSol$ are the generalized coordinates of $\stateROMSol$ and $\stateROMStarSol$ and it is additionally noted that $\norm{\basisspaceArg{n}} = 1$ (orthonormal matrix).
%Critically, we now utilize the fact that the system governing \methodAcronym\ with \spatialAcronym\ trial subspaces is Hamiltonian. Applying Liouville's theorem for Hamiltonian systems, we have the requirement 
%%states that, for Hamiltonian systems, ``the volume enclosed by a closed surface in phase-space is 
%%constant as that volume moves through phase-space~\cite{liouville}.'' In the present context, Liouville's theorem states, 
%$$ \frac{d}{dt}  \sqrt{ \norm{\genstateROMSolArg{n} - \genstateROMStarSolArg{n}}^2 + \norm{\adjointROMSolArg{n} - 
%\adjointROMStarSolArg{n} }^2}  = \bz \qquad t \in [\timeStartArg{n},\timeEndArg{n}].$$
%Thus, we have
%\begin{align*}
%\intSlabArg{n} \norm{\stateROMSolArgt{n}{t} - \stateROMStarSolArgt{n}{t}} dt &\le 
%\intSlabArg{n} \norm{\genstateROMSolArgt{n}{t} - \genstateROMStarSolArgt{n}{t}} + \norm{ \adjointROMSolArgt{n}{t} - 
% \adjointROMStarSolArgt{n}{t} } dt \\
%&\le  \DeltaSlabArg{n}  \sqrt{ \norm{\genstateROMSolArgt{n}{\timeStartArg{n}} - \genstateROMStarSolArgt{n}{\timeStartArg{n} }}^2 + \norm{\adjointROMSolArgt{n}{\timeStartArg{n}} - 
%\adjointROMStarSolArgt{n}{\timeStartArg{n}}  }^2}.
%\end{align*}
%================
Using the definition of the costate~\eqref{eq:costate_def} and residual, 
\begin{multline*}
\intSlabArg{n} \norm{\stateROMSolArgt{n}{t} - \stateROMStarSolArgt{n}{t}} dt 
\le \\  \DeltaSlabArg{n} \big[ \sqrt{ \norm{\genstateROMSolArgt{n}{\timeStartArg{n}} - \genstateROMStarSolArgt{n}{\timeStartArg{n} }}^2 +   \norm{\basisspaceTArg{n} \resid( \basisspaceArg{n} \stateROMSolArg{n} + \stateInterceptArg{n} , {\timeStartArg{n}} ) - 
 \basisspaceTArg{n} \resid( \basisspaceArg{n} \stateROMStarSolArg{n} + \stateInterceptArg{n}, {\timeStartArg{n}}  )}^2  } \big].
\end{multline*}
As, 
\begin{multline*}
\norm{\basisspaceTArg{n} \resid( \basisspace \stateROMSolArg{n} + \stateInterceptArg{n} , {\timeStartArg{n}}) - 
 \basisspaceTArg{n} \resid( \basisspaceArg{n} \stateROMStarSolArg{n} + \stateInterceptArg{n}, {\timeStartArg{n}}  )}
\le \\
 \norm{\basisspaceTArg{n}} \norm{ \resid( \basisspaceArg{n} \stateROMSolArg{n} + \stateInterceptArg{n} , {\timeStartArg{n}}) - 
  \resid( \basisspaceArg{n} \stateROMStarSolArg{n} + \stateInterceptArg{n} , {\timeStartArg{n}}  )},
\end{multline*}
with $\norm{\basisspaceTArg{n}} = 1$, applying A1 yields
%$$\norm{\stateROMSolArg{n} - \stateROMStarSolArg{n}} \le  \norm{\stateROMSolArgt{n}{\timeStartArg{n}} - \stateROMStarSolArgt{n}{\timeStartArg{n}}} + \norm{\resid( \basisspace \stateROMSolArgt{n}{\timeStartArg{n}} + \stateIntercept ) - 
% \resid( \basisspace \stateROMStarSolArgt{n}{\timeStartArg{n}}  + \stateIntercept )} .$$
%Applying A1,
%\begin{equation*}
%\intSlabArg{n} \norm{\stateROMSolArgt{n}{t} - \stateROMStarSolArgt{n}{t}} dt 
%\le  \DeltaSlabArg{n} \big[ \sqrt{ \norm{\genstateROMSolArgt{n}{\timeStartArg{n}} - \genstateROMStarSolArgt{n}{\timeStartArg{n} }}^2 +   \norm{\resid( \basisspace \stateROMSolArg{n} + \stateIntercept , {\timeStartArg{n}} ) - 
%\resid( \basisspace \stateROMStarSolArg{n} + \stateIntercept, {\timeStartArg{n}}  )}^2  } \big].
%\end{equation*}
%\begin{equation*}
%\intSlabArg{n} \norm{\stateROMSolArgt{n}{t} - \stateROMStarSolArgt{n}{t}} dt 
%\le  \DeltaSlabArg{n} \big[ \sqrt{ \norm{\genstateROMSolArgt{n}{\timeStartArg{n}} - \genstateROMStarSolArgt{n}{\timeStartArg{n} }}^2 +  \lipshitz^2 \norm{ \stateROMSolArg{n} - 
% \stateROMStarSolArg{n}}^2  } \big].
%\end{equation*}
\begin{equation}\label{eq:ustarbound}
\intSlabArg{n} \norm{\stateROMSolArgt{n}{t} - \stateROMStarSolArgt{n}{t}} dt 
\le  \DeltaSlabArg{n} \sqrt{1 + \lipshitz} \norm{\stateROMSolArgt{n}{\timeStartArg{n}} - \stateROMStarSolArgt{n}{\timeStartArg{n} } }.
\end{equation}
%Lastly, as $\basisspace$ is orthonormal, 
%\begin{align*}
%\norm{ \stateROMSolArgt{n}{\timeStartArg{n}}  -   \stateROMStarSolArgt{n}{\timeStartArg{n}}} &= 
%\norm{ \big( \basisspace \genstateROMSolArgt{n}{\timeStartArg{n}} + \stateIntercept \big) - \big( \basisspace \genstateROMStarSolArgt{n}{\timeStartArg{n}} + \stateIntercept \big) }, \\ 
%&= \norm{  \basisspace \big( \genstateROMSolArgt{n}{\timeStartArg{n}}  -   \genstateROMStarSolArgt{n}{\timeStartArg{n}} \big) } ,\\
%&=  \norm{ \genstateROMSolArgt{n}{\timeStartArg{n}}  -   \genstateROMStarSolArgt{n}{\timeStartArg{n}}}.
%\end{align*}
%Thus,
%\begin{equation}\label{eq:ustarbound}
%\intSlabArg{n} \norm{\stateROMSolArgt{n}{t} - \stateROMStarSolArgt{n}{t}} dt 
%\le  \DeltaSlabArg{n} (1 + \lipshitz) \norm{\stateROMSolArgt{n}{\timeStartArg{n}} - \stateROMStarSolArgt{n}{\timeStartArg{n} } }.
%\end{equation}
Substituting~\eqref{eq:ustarbound} into~\eqref{eq:boundtmp}, noting that $\stateROMStarSolArgt{n}{\timeStartArg{n}}=  \stateFOMArg{n}{\timeStartArg{n}}$,  and using $\errorArgt{n}{\timeStartArg{n}} =  \stateFOMArg{n}{\timeStartArg{n} } -  \stateROMSolArgt{n}{\timeStartArg{n}}$ gives the upper bound
\begin{equation*}
\intSlabArg{n} \norm{\errorArgt{n}{t}} dt \le \DeltaSlabArg{n} \sqrt{1 + \lipshitz} \norm{ \errorArgt{n}{\timeStartArg{n}}  }   + \lipshitziArg{n} \intSlabArg{n} \norm{ \resid(\stateFOMProjSolArgt{n}{t},t) } dt.
\end{equation*}
%\begin{equation*}
%\intSlabArg{n} \norm{\errorArg{n}} dt \le \DeltaSlabArg{} (1 + \lipshitz) \norm{\errorArgt{n}{\timeStartArg{n}} }   + \lipshitzi \intSlabArg{n} \norm{ \resid(\stateFOMProjSolArg{n}) } dt.
%\end{equation*}
%Summing errors over all time slabs and using $\errorArgt{n}{\timeStartArg{n}} = \stateROMSolArgt{n}{\timeStartArg{n}} - \stateROMStarSolArgt{n}{\timeStartArg{n} }$,
%\begin{equation*}
%\sum_{i=1}^{\nslabs} \intSlabArg{i} \norm{\errorArgt{i}{t}} dt  \le \sum_{i=1}^{\nslabs} \DeltaSlabArg{i} (1 + \lipshitz) \norm{\errorArgt{i}{\timeStartArg{i}} }   + \lipshitziArg{n} \int_0^T \norm{ \resid(\stateFOMProjSolArgt{n}{t}) } dt.
%\end{equation*}
%\EP{Can bound error at the start of the time-slab vs integral over previous time-slab; but not sure if I like this}
%===============
\end{proof}

\begin{corollary}
When only a single space-time window is used, error bounds are given by
\begin{equation*}
\int_0^T \norm{\errorArgt{1}{t}} dt \le \lipshitziArg{1} \int_0^T \norm{\resid(\stateFOMProjSolArgt{1}{t},t)}dt. %+
% T \sqrt{1 + \lipshitz} \norm{\errorArgt{1}{\timeStartArg{1}} } .
\end{equation*}
\end{corollary}
\begin{proof}
Setting $n=1$ in~\eqref{eq:apriori_bound} with the time intervals $\timeStartArg{1}=0$, $\timeEndArg{1}=T$, employing assumption A3, and noting that the initial conditions are known yields the desired result.
\end{proof}

\begin{comment}
\begin{theorem}(\textit{A priori} error bounds for space--time trial subspace)\label{theorem:apriori_bound_st}
Error bounds for \methodAcronymROMs\ with a space--time trial subspace are:
\begin{equation}\label{eq:apriori_bound}
\intSlabArg{n} \norm{\errorArgt{n}{t}} dt \le \sum_{n=1}^{\nslabs} \DeltaSlabArg{n} \norm{\errorArgt{n}{\timeStartArg{n}} }   + \lipshitzi \int_0^T \norm{ \resid(\stateFOMProjSolArgt{n}{t}) } dt.
\end{equation}
\EP{These are now wrong if the norm of the trial basis isn't constant for each time-instance.}
\end{theorem}

\begin{proof}
We note that $\stgenstateArg{n}$ are the space--time generalized coordinates implicitly defined as the solution to the \methodAcronym\ minimization problem~\eqref{eq:obj_gen_slab_spacetime}. Additionally, to account for the 
fact that the initial conditions into the $n$th window can be incorrect, we define a new quantity $\stgenstateStarArg{n}$ to implicitly be the solution to the minimization problem with the correct initial conditions,
\begin{align*}
& \stgenstateStarArg{n} \defeq \underset{\stgenstatey}{\text{arg min } } \objectiveArg{n}(\stbasisArgnt{n} \stgenstatey + \stateInterceptArg{n}),\\
& \stbasisArg{n}{\timeStartArg{n}} \stgenstateStarArg{n} + \stateIntercept = \stateFOMSolArgt{n}{\timeStartArg{n}}.
\end{align*}
Next, the results presented in Theorem~\ref{theorem:apriori_bound} hold for the space--time trial subspace through bound~\eqref{eq:boundtmp}. We now derive an upper bound for $\intSlabArg{n} \norm{ \stateROMSolArgt{n}{t} - \stateROMStarSolArgt{n}{t}}$ for the space--time case. By definition of the space--time trial subspace,
\begin{align*}
 \norm{ \stateROMSolArgt{n}{t} - \stateROMStarSolArgt{n}{t}} & \equiv \norm{ \stbasisArg{n}{t}\big[ \stgenstateArg{n} - \stgenstateStarArg{n} \big] }\\
& \equiv \norm{ \big[ \stgenstateArg{n} - \stgenstateStarArg{n} \big] }.
\end{align*}
As $ \stgenstateArg{n} - \stgenstateStarArg{n} $ contains no temporal dependence, it then holds that,
$$\frac{d}{dt}  \norm{ \stateROMSolArgt{n}{t} - \stateROMStarSolArgt{n}{t}} \equiv 0.$$
Thus we have,
$$ \intSlabArg{n} \norm{ \stateROMSolArgt{n}{t} - \stateROMStarSolArgt{n}{t}} = \DeltaSlabArg{n} \errorArgt{n}{\timeStartArg{n}}.$$
Inserting the above into~\eqref{eq:boundtmp},
\begin{equation*}
\intSlabArg{n} \norm{\errorArgt{n}{t}} dt \le \DeltaSlabArg{n} \norm{\errorArgt{n}{\timeStartArg{n}}  }   + \lipshitzi \intSlabArg{n} \norm{ \resid(\stateFOMProjSolArgt{n}{t}) } dt.
\end{equation*}
%Summing over all time slabs,
%\begin{equation*}
%\intSlabArg{n} \norm{\errorArg{}} dt \le \sum_{i=1}^{\nslabs} \DeltaSlabArg{i}  \norm{\errorArgt{i}{\timeStartArg{i}} }   + \lipshitzi \int_0^T \norm{ \resid(\stateFOMProjSolArgt{n}{t}) } dt.
%\end{equation*}
\end{proof}
\end{comment}
\subsection{Discussion} 
Theorem~\ref{theorem:apriori_bound} provides \textit{a priori} bounds on the integrated normed error for \methodAcronym\ employing \spatialAcronym\ trial subspaces. We make several 
observations. First, it is observed \methodAcronym\ is subject to recursive error bounds (through the first term on the RHS in the upper bound~\eqref{eq:apriori_bound}). As the number of time windows grows, so does the recursive growth of error. Second, we observe that when a single window spans the entire domain, the error in the \methodAcronym\ with \spatialAcronym\ trial subspaces is bounded \textit{a priori} by the residual of the projected FOM solution. 
%\begin{theorem}\label{theorem:aposteriori}
%A posteriori error bounds on the the CST-LSPG ROM are,
%%The a posteriori error bound is then,
%\begin{equation}\label{eq:th_1}
%\int_0^t \norm{\error(\tau)} d\tau \le \alpha \int_0^t \norm{ \resid(\approxstate_R(\tau)) } d\tau.
%\end{equation}
%\end{theorem}
%
%\begin{proof}
%Invoking the assumption of inverse Lipshitz continuity,
%$$\norm{ \resid(\state_f) - \resid(\approxstate_R) } \ge \frac{1}{\alpha} \norm{\state_f - \approxstate_R} $$
%Integrating from $0$ to $t$,
%$$\int_0^t \norm{ \resid(\state_f(\tau) ) - \resid(\approxstate_R(\tau)) } d\tau \ge \frac{1}{\alpha} \int_0^t \norm{\state_f(\tau) - \approxstate_R(s)} d\tau.$$
%Multiplying by $\alpha$ and noting that $\resid(\state(\tau)) = \boldsymbol 0$,
%$$  \int_0^t \norm{\error(\tau)} d\tau \le \alpha \int_0^t \norm{ \resid(\approxstate_R(\tau)) } d\tau$$
%\end{proof}
%Although Theorem~\ref{theorem:aposteriori} provides an upper bound on the error, it is not particularly insightful as it is not clear what the magnitude of the residual may be.

%\begin{theorem}\label{theorem:apriori}
%\textit{A priori} error bounds for the CST-LSPG ROM are,
%$$\int_0^t \norm{\error(\tau)}^2 d\tau \le \alpha \int_0^t \norm{ \resid( \Pi \state_f(\tau)) }^2 d\tau,$$
%where $\Pi$ is the projector onto the trial-space,
%$$\Pi = \basismat \basismat^T.$$
%\end{theorem}
%
%\begin{proof}
%Invoking the assumption of inverse Lipshitz continuity,
%$$\norm{ \resid(\state_f) - \resid(\approxstate_R) }^2 \ge \frac{1}{\alpha} \norm{\state_f - \approxstate_R}^2 $$
%Integrating from $0$ to $t$,
%$$\int_0^t \norm{ \resid(\state_f(\tau) ) - \resid(\approxstate_R(\tau)) }^2 d\tau \ge \frac{1}{\alpha} \int_0^t \norm{\state_f(\tau) - \approxstate_R(\tau)}^2 d\tau.$$
%Multiplying by $\alpha$ and noting that $\resid(\state_f(\tau)) = \boldsymbol 0$,
%$$  \int_0^t \norm{\error(\tau)}^2 d\tau \le \alpha \int_0^t \norm{ \resid(\approxstate_R(\tau)) }^2 d\tau $$
%The TC-LSPG ROM satisfies the optimality condition,
%%$$\approxstate =\underset{\statey \in \stspace }{\text{argmin}}\frac{1}{2} \int_0^t \big[ \dot{ \statey} - \velocity(\statey,\tau;\param) \big]^T \stweightingMat(\tau) \big[ \dot{ \statey} - \velocity(\statey,\tau;\param) \big] d\tau .$$
%$$\approxstate_R =\underset{\statey \in \stspace }{\text{argmin}}\frac{1}{2} \int_0^t \norm{ \resid(\statey(\tau)) }^2 d\tau .$$
%Therefore,
%$$ \int_0^t \norm{ \resid(\approxstate_R(\tau)) }^2 d\tau \le \int_0^t \norm{ \resid( \Pi \state_f(\tau)) }^2 d\tau .$$
%It then follows that,
%$$  \int_0^t \norm{\error(\tau)}^2 d\tau \le \alpha \int_0^t \norm{ \resid(\approxstate_R(\tau)) }^2 d\tau \le  \alpha \int_0^t \norm{ \resid( \Pi \state(\tau)) }^2 d\tau.$$
%\end{proof}
%Theorem~\ref{theorem:apriori} shows that the error in the CST-LSPG ROM is bounded \textit{a priori} by the integrated residual of the projected truth solution. This condition provides a a strong statement about the error in the CST-LSPG ROM. It shows that the upper bound on the error is solely based on the Lipshitz constant and the projection error. Error bounds for the Galekrin ROM, on the other hand, are known to grow exponentially in time.
%
